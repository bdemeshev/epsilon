\documentclass[11pt,russian,]{article}
\usepackage{lmodern}
\usepackage{amssymb,amsmath}
\usepackage{ifxetex,ifluatex}
\usepackage{fixltx2e} % provides \textsubscript
\ifnum 0\ifxetex 1\fi\ifluatex 1\fi=0 % if pdftex
  \usepackage[T1]{fontenc}
  \usepackage[utf8]{inputenc}
\else % if luatex or xelatex
  \ifxetex
    \usepackage{mathspec}
  \else
    \usepackage{fontspec}
  \fi
  \defaultfontfeatures{Ligatures=TeX,Scale=MatchLowercase}
    \setmainfont[]{Arial}
\fi
% use upquote if available, for straight quotes in verbatim environments
\IfFileExists{upquote.sty}{\usepackage{upquote}}{}
% use microtype if available
\IfFileExists{microtype.sty}{%
\usepackage{microtype}
\UseMicrotypeSet[protrusion]{basicmath} % disable protrusion for tt fonts
}{}
\usepackage[left=2cm, right=2cm, top=2cm, bottom=2cm]{geometry}
\usepackage{hyperref}
\PassOptionsToPackage{usenames,dvipsnames}{color} % color is loaded by hyperref
\hypersetup{unicode=true,
            pdftitle={Ура!},
            pdfauthor={Винни-Пух :)},
            colorlinks=true,
            linkcolor=blue,
            citecolor=blue,
            urlcolor=blue,
            breaklinks=true}
\urlstyle{same}  % don't use monospace font for urls
\ifnum 0\ifxetex 1\fi\ifluatex 1\fi=0 % if pdftex
  \usepackage[shorthands=off,main=russian]{babel}
\else
  \usepackage{polyglossia}
  \setmainlanguage[]{russian}
\fi
\usepackage{graphicx,grffile}
\makeatletter
\def\maxwidth{\ifdim\Gin@nat@width>\linewidth\linewidth\else\Gin@nat@width\fi}
\def\maxheight{\ifdim\Gin@nat@height>\textheight\textheight\else\Gin@nat@height\fi}
\makeatother
% Scale images if necessary, so that they will not overflow the page
% margins by default, and it is still possible to overwrite the defaults
% using explicit options in \includegraphics[width, height, ...]{}
\setkeys{Gin}{width=\maxwidth,height=\maxheight,keepaspectratio}
\IfFileExists{parskip.sty}{%
\usepackage{parskip}
}{% else
\setlength{\parindent}{0pt}
\setlength{\parskip}{6pt plus 2pt minus 1pt}
}
\setlength{\emergencystretch}{3em}  % prevent overfull lines
\providecommand{\tightlist}{%
  \setlength{\itemsep}{0pt}\setlength{\parskip}{0pt}}
\setcounter{secnumdepth}{5}
% Redefines (sub)paragraphs to behave more like sections
\ifx\paragraph\undefined\else
\let\oldparagraph\paragraph
\renewcommand{\paragraph}[1]{\oldparagraph{#1}\mbox{}}
\fi
\ifx\subparagraph\undefined\else
\let\oldsubparagraph\subparagraph
\renewcommand{\subparagraph}[1]{\oldsubparagraph{#1}\mbox{}}
\fi

%%% Use protect on footnotes to avoid problems with footnotes in titles
\let\rmarkdownfootnote\footnote%
\def\footnote{\protect\rmarkdownfootnote}

%%% Change title format to be more compact
\usepackage{titling}

% Create subtitle command for use in maketitle
\newcommand{\subtitle}[1]{
  \posttitle{
    \begin{center}\large#1\end{center}
    }
}

\setlength{\droptitle}{-2em}
  \title{Ура!}
  \pretitle{\vspace{\droptitle}\centering\huge}
  \posttitle{\par}
  \author{Винни-Пух :)}
  \preauthor{\centering\large\emph}
  \postauthor{\par}
  \predate{\centering\large\emph}
  \postdate{\par}
  \date{10/9/2017}

\newfontfamily{\cyrillicfonttt}{Arial}
\newfontfamily{\cyrillicfont}{Arial}
\newfontfamily{\cyrillicfontsf}{Arial}

\begin{document}
\maketitle

Истинная зависимость имеет вид \[
y_i = \beta_1 + \beta_x x_i + \beta_w w_i + u_i
\]

Ошибка \(u_i\) некоррелирована с регрессорами \(x_i\) и \(w_i\).

Проблема в том, что \(w_i\) не наблюдаемы.

Оценить \(\beta_w\) мы не надеемся! Мы думаем, есть ли возможность
оценить состоятельно хотя бы \(\beta_x\).

Есть два способа!

Способ 1. Раздобыли \(proxy\) для \(w\). Требования к прокси-переменной.

Для удобства записи требований временно предположим, что мы разложили
пропущенную переменную на ту часть, которая связана с прокси, и часть,
некоррелированную с прокси:

\[
w_i = \delta_1 + \delta_p proxy_i + \nu_i
\]

\begin{enumerate}
\def\labelenumi{\arabic{enumi}.}
\item
  Прокси коррелированна с пропущенной переменной,
  \(Cov(proxy_i, w_i) \neq 0\) или \(\delta_p \neq 0\).
\item
  Ошибка \(u_i\) некоррелирована с прокси \(proxy_i\)
\item
  Ошибка \(\nu_i\) некоррелирована с прокси \(proxy_i\) и \(x_i\).
\end{enumerate}

Использование прокси переменной. Строим банальную регрессию \(y_i\) на
константу, \(x_i\) и \(proxy_i\). При этом \(\hat\beta_x\) состоятельная
для \(\beta_x\), а \(\hat\beta_1\) не состоятельна для \(\beta_1\) и
\(\hat\beta_{proxy}\) несостоятельная для \(\beta_w\).

Способ 2. Раздобыли инструмент \(inst\) для \(x\). Требования к
инструментальной переменной.

\begin{enumerate}
\def\labelenumi{\arabic{enumi}.}
\item
  Инструментальная переменная коррелирована с включенным регрессором,
  \(Cov(inst, x) \neq 0\).
\item
  Инструментальная переменная некоррелирована с хвостиком
  \(\beta_w w_i + u_i\), \(Cov(inst, \beta_w w_i + u_i)=0\).
\end{enumerate}

Использование инструментальной переменной. Двухшаговый МНК. Строим
регрессию \(x\) на константу и \(inst\). Получаем \(\hat x\). Строим
регрессию \(y\) на константу и \(\hat x\). При этом \(\hat\beta_x\)
состоятельная для \(\beta_x\).


\end{document}
