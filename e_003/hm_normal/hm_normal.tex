\documentclass[11pt,russian,]{article}
\usepackage{lmodern}
\usepackage{amssymb,amsmath}
\usepackage{ifxetex,ifluatex}
\usepackage{fixltx2e} % provides \textsubscript
\ifnum 0\ifxetex 1\fi\ifluatex 1\fi=0 % if pdftex
  \usepackage[T1]{fontenc}
  \usepackage[utf8]{inputenc}
\else % if luatex or xelatex
  \ifxetex
    \usepackage{mathspec}
  \else
    \usepackage{fontspec}
  \fi
  \defaultfontfeatures{Ligatures=TeX,Scale=MatchLowercase}
    \setmainfont[]{Linux Libertine O}
    \setsansfont[]{Linux Libertine O}
    \setmonofont[Mapping=tex-ansi]{Linux Libertine O}
\fi
% use upquote if available, for straight quotes in verbatim environments
\IfFileExists{upquote.sty}{\usepackage{upquote}}{}
% use microtype if available
\IfFileExists{microtype.sty}{%
\usepackage{microtype}
\UseMicrotypeSet[protrusion]{basicmath} % disable protrusion for tt fonts
}{}
\usepackage[margin=1in]{geometry}
\usepackage{hyperref}
\hypersetup{unicode=true,
            pdftitle={Хершел-Максвелл и нормальное распределение},
            pdfauthor={Винни-Пух},
            pdfborder={0 0 0},
            breaklinks=true}
\urlstyle{same}  % don't use monospace font for urls
\ifnum 0\ifxetex 1\fi\ifluatex 1\fi=0 % if pdftex
  \usepackage[shorthands=off,main=russian]{babel}
\else
  \usepackage{polyglossia}
  \setmainlanguage[]{russian}
\fi
\usepackage[style=alphabetic]{biblatex}

\usepackage{graphicx,grffile}
\makeatletter
\def\maxwidth{\ifdim\Gin@nat@width>\linewidth\linewidth\else\Gin@nat@width\fi}
\def\maxheight{\ifdim\Gin@nat@height>\textheight\textheight\else\Gin@nat@height\fi}
\makeatother
% Scale images if necessary, so that they will not overflow the page
% margins by default, and it is still possible to overwrite the defaults
% using explicit options in \includegraphics[width, height, ...]{}
\setkeys{Gin}{width=\maxwidth,height=\maxheight,keepaspectratio}
\IfFileExists{parskip.sty}{%
\usepackage{parskip}
}{% else
\setlength{\parindent}{0pt}
\setlength{\parskip}{6pt plus 2pt minus 1pt}
}
\setlength{\emergencystretch}{3em}  % prevent overfull lines
\providecommand{\tightlist}{%
  \setlength{\itemsep}{0pt}\setlength{\parskip}{0pt}}
\setcounter{secnumdepth}{0}
% Redefines (sub)paragraphs to behave more like sections
\ifx\paragraph\undefined\else
\let\oldparagraph\paragraph
\renewcommand{\paragraph}[1]{\oldparagraph{#1}\mbox{}}
\fi
\ifx\subparagraph\undefined\else
\let\oldsubparagraph\subparagraph
\renewcommand{\subparagraph}[1]{\oldsubparagraph{#1}\mbox{}}
\fi

%%% Use protect on footnotes to avoid problems with footnotes in titles
\let\rmarkdownfootnote\footnote%
\def\footnote{\protect\rmarkdownfootnote}

%%% Change title format to be more compact
\usepackage{titling}

% Create subtitle command for use in maketitle
\newcommand{\subtitle}[1]{
  \posttitle{
    \begin{center}\large#1\end{center}
    }
}

\setlength{\droptitle}{-2em}
  \title{Хершел-Максвелл и нормальное распределение}
  \pretitle{\vspace{\droptitle}\centering\huge}
  \posttitle{\par}
  \author{Винни-Пух}
  \preauthor{\centering\large\emph}
  \postauthor{\par}
  \predate{\centering\large\emph}
  \postdate{\par}
  \date{2017-12-01}

\newfontfamily{\cyrillicfonttt}{Linux Libertine O}
\newfontfamily{\cyrillicfont}{Linux Libertine O}
\newfontfamily{\cyrillicfontsf}{Linux Libertine O}
\newcommand{\E}{\mathbb{E}}
\newcommand{\Var}{\mathbb{V}ar}

\begin{document}
\maketitle

{
\setcounter{tocdepth}{2}
\tableofcontents
}
\subsection{Молекулы газа}\label{-}

В замкнутом загончике на плоскости хаотично движутся молекулы газа. Мы
ловим одну из них случайно и измеряем вектор скоростей \[
V = \begin{pmatrix}
V_x \\
V_y \\
\end{pmatrix}.
\]

Максвелл предположил, что

М1. Если мы повернём нашу картинку на произвольный угол и повторим
измерения, то закон распределения нового вектора \(V'\) будет совпадать
с законом распределения вектора \(V\).

М2. Если мы знаем горизонтальную составляющую скорости, то это не даёт
нам никакой информации о вертикальной составляющей, то есть случайные
величины \(V_x\) и \(V_y\) независимы.

Заметим, что единицы измерения скорости мы можем выбираться произвольно,
поэтому давайте дополним предположения Максвелла предположением

М3. Единицы измерения скорости выбраны так, что \(\Var(V_x)=1\).

\subsection{Первый поворот}\label{-}

Помимо горизонтальной и вертикальной составлящих вектора скорости,
\(V_x\) и \(V_y\), рассмотрим ещё две величины, \(U\) --- угол с
горизонтальной осью и \(R=\sqrt{V_x^2+V_y^2}\) --- скалярную скорость,
длину вектора скорости.

Естественно, \(V_x = R\cos U\), \(V_y = R\sin U\).

\begin{itemize}
\tightlist
\item
  Какой вектор получится, если вектор \(V\) повернуть на \(90^{\circ}\)
  против часовой стрелки?
\end{itemize}

Получится вектор \[
V'=\begin{pmatrix}
-V_y \\
V_x \\
\end{pmatrix}
\]

По предпосылке М1 вектор \(V'\) должен иметь такое же распределение, как
вектор \(V\).

\begin{itemize}
\tightlist
\item
  Чему равны \(\E(V_x)\), \(\E(V_y)\), \(\Var(V_y)\)?
\end{itemize}

Раз уж \(V' \sim V\), то \(-V_y \sim V_x\) и \(V_x \sim V_y\). Значит
\(\E(-V_y) = \E(V_x)\), и одновременно \(\E(V_x) = \E(V_y)\). Это
возможно только в случае \(\E(V_x) = \E(V_y) = 0\).

Строго говоря, осталась ещё возможность, что математическое ожидание не
существует.

Аналогично, \(\Var(V_x) = \Var(V_y)\) и по предпосылке М3
\(\Var(V_x) = \Var(V_y) = 1\).

\begin{itemize}
\tightlist
\item
  Как распределена величина \(U\)?
\end{itemize}

Заметим, что при повороте на произвольный угол \(\alpha\), этот угол
прибавляется к величине \(U\). Если при этом сумма выйдет за \(2\pi\),
то нужно ещё и вычесть \(2\pi\). По предпосылке М1 функция плотности
\(U\) может быть только постоянной, \(f(u+\alpha)=f(u)\) при
\(0\leq u +\alpha < 2\pi\).

Значит \(U\) распределена равномерно на \([0;2\pi)\) и её функция
плотности равна \[
f(u) = \begin{cases}
\frac{1}{2\pi}, \text{ если } u \in [0;2\pi) \\
0, \text{ иначе} \\
\end{cases}
\]

\printbibliography


\end{document}
