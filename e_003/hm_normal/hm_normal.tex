\documentclass[11pt,russian,]{article}
\usepackage{lmodern}
\usepackage{amssymb,amsmath}
\usepackage{ifxetex,ifluatex}
\usepackage{fixltx2e} % provides \textsubscript
\ifnum 0\ifxetex 1\fi\ifluatex 1\fi=0 % if pdftex
  \usepackage[T1]{fontenc}
  \usepackage[utf8]{inputenc}
\else % if luatex or xelatex
  \ifxetex
    \usepackage{mathspec}
  \else
    \usepackage{fontspec}
  \fi
  \defaultfontfeatures{Ligatures=TeX,Scale=MatchLowercase}
    \setmainfont[]{Linux Libertine O}
    \setsansfont[]{Linux Libertine O}
    \setmonofont[Mapping=tex-ansi]{Linux Libertine O}
\fi
% use upquote if available, for straight quotes in verbatim environments
\IfFileExists{upquote.sty}{\usepackage{upquote}}{}
% use microtype if available
\IfFileExists{microtype.sty}{%
\usepackage{microtype}
\UseMicrotypeSet[protrusion]{basicmath} % disable protrusion for tt fonts
}{}
\usepackage[margin=1in]{geometry}
\usepackage{hyperref}
\hypersetup{unicode=true,
            pdftitle={Хершел-Максвелл и нормальное распределение},
            pdfauthor={Винни-Пух},
            pdfborder={0 0 0},
            breaklinks=true}
\urlstyle{same}  % don't use monospace font for urls
\ifnum 0\ifxetex 1\fi\ifluatex 1\fi=0 % if pdftex
  \usepackage[shorthands=off,main=russian]{babel}
\else
  \usepackage{polyglossia}
  \setmainlanguage[]{russian}
\fi
\usepackage[style=alphabetic]{biblatex}

\usepackage{graphicx,grffile}
\makeatletter
\def\maxwidth{\ifdim\Gin@nat@width>\linewidth\linewidth\else\Gin@nat@width\fi}
\def\maxheight{\ifdim\Gin@nat@height>\textheight\textheight\else\Gin@nat@height\fi}
\makeatother
% Scale images if necessary, so that they will not overflow the page
% margins by default, and it is still possible to overwrite the defaults
% using explicit options in \includegraphics[width, height, ...]{}
\setkeys{Gin}{width=\maxwidth,height=\maxheight,keepaspectratio}
\IfFileExists{parskip.sty}{%
\usepackage{parskip}
}{% else
\setlength{\parindent}{0pt}
\setlength{\parskip}{6pt plus 2pt minus 1pt}
}
\setlength{\emergencystretch}{3em}  % prevent overfull lines
\providecommand{\tightlist}{%
  \setlength{\itemsep}{0pt}\setlength{\parskip}{0pt}}
\setcounter{secnumdepth}{0}
% Redefines (sub)paragraphs to behave more like sections
\ifx\paragraph\undefined\else
\let\oldparagraph\paragraph
\renewcommand{\paragraph}[1]{\oldparagraph{#1}\mbox{}}
\fi
\ifx\subparagraph\undefined\else
\let\oldsubparagraph\subparagraph
\renewcommand{\subparagraph}[1]{\oldsubparagraph{#1}\mbox{}}
\fi

%%% Use protect on footnotes to avoid problems with footnotes in titles
\let\rmarkdownfootnote\footnote%
\def\footnote{\protect\rmarkdownfootnote}

%%% Change title format to be more compact
\usepackage{titling}

% Create subtitle command for use in maketitle
\newcommand{\subtitle}[1]{
  \posttitle{
    \begin{center}\large#1\end{center}
    }
}

\setlength{\droptitle}{-2em}
  \title{Хершел-Максвелл и нормальное распределение}
  \pretitle{\vspace{\droptitle}\centering\huge}
  \posttitle{\par}
  \author{Винни-Пух}
  \preauthor{\centering\large\emph}
  \postauthor{\par}
  \predate{\centering\large\emph}
  \postdate{\par}
  \date{2017-12-24}

\newfontfamily{\cyrillicfonttt}{Linux Libertine O}
\newfontfamily{\cyrillicfont}{Linux Libertine O}
\newfontfamily{\cyrillicfontsf}{Linux Libertine O}
\newcommand{\E}{\mathbb{E}}
\newcommand{\Var}{\mathbb{V}ar}
\renewcommand{\P}{\mathbb{P}}

\begin{document}
\maketitle

{
\setcounter{tocdepth}{2}
\tableofcontents
}
\subsection{Молекулы газа}\label{-}

В замкнутом загончике на плоскости хаотично движутся молекулы газа. Мы
ловим одну из них случайно и измеряем вектор скоростей \[
V = \begin{pmatrix}
V_x \\
V_y \\
\end{pmatrix}.
\]

Максвелл предположил, что

М1. Если мы повернём нашу картинку на произвольный угол и повторим
измерения, то закон распределения нового вектора \(V'\) будет совпадать
с законом распределения вектора \(V\).

М2. Если мы знаем горизонтальную составляющую скорости, то это не даёт
нам никакой информации о вертикальной составляющей, то есть случайные
величины \(V_x\) и \(V_y\) независимы.

Заметим, что единицы измерения скорости мы можем выбираться произвольно,
поэтому давайте дополним предположения Максвелла предположением

М3. Единицы измерения скорости выбраны так, что \(\Var(V_x)=1\).

\subsection{Первый поворот}\label{-}

Помимо горизонтальной и вертикальной составлящих вектора скорости,
\(V_x\) и \(V_y\), рассмотрим ещё две величины, \(U\) --- угол с
горизонтальной осью и \(R=\sqrt{V_x^2+V_y^2}\) --- скалярную скорость,
длину вектора скорости.

Естественно, \(V_x = R\cos U\), \(V_y = R\sin U\).

\begin{itemize}
\tightlist
\item
  Какой вектор получится, если вектор \(V\) повернуть на \(90^{\circ}\)
  против часовой стрелки?
\end{itemize}

Получится вектор \[
V'=\begin{pmatrix}
-V_y \\
V_x \\
\end{pmatrix}
\]

По предпосылке М1 вектор \(V'\) должен иметь такое же распределение, как
вектор \(V\).

\begin{itemize}
\tightlist
\item
  Чему равны \(\E(V_x)\), \(\E(V_y)\), \(\Var(V_y)\)?
\end{itemize}

Раз уж \(V' \sim V\), то \(-V_y \sim V_x\) и \(V_x \sim V_y\). Значит
\(\E(-V_y) = \E(V_x)\), и одновременно \(\E(V_x) = \E(V_y)\). Это
возможно только в случае \(\E(V_x) = \E(V_y) = 0\).

Строго говоря, осталась ещё возможность, что математическое ожидание не
существует.

Аналогично, \(\Var(V_x) = \Var(V_y)\) и по предпосылке М3
\(\Var(V_x) = \Var(V_y) = 1\).

\begin{itemize}
\tightlist
\item
  Как распределена величина \(U\)?
\end{itemize}

Заметим, что при повороте на произвольный угол \(\alpha\), этот угол
прибавляется к величине \(U\). Если при этом сумма выйдет за \(2\pi\),
то нужно ещё и вычесть \(2\pi\). По предпосылке М1 функция плотности
\(U\) может быть только постоянной, \(f(u+\alpha)=f(u)\) при
\(0\leq u +\alpha < 2\pi\).

Значит \(U\) распределена равномерно на \([0;2\pi)\) и её функция
плотности равна \[
f(u) = \begin{cases}
\frac{1}{2\pi}, \text{ если } u \in [0;2\pi) \\
0, \text{ иначе} \\
\end{cases}
\]

\subsection{\texorpdfstring{Вид функции плотности
\(f(v_x, v_y)\)}{Вид функции плотности f(v\_x, v\_y)}}\label{---fv_x-v_y}

\begin{itemize}
\tightlist
\item
  Какой вид имеет совместная функция плотности \(f(v_x, v_y)\)?
\end{itemize}

По предпосылке М1 совместная функция плотности может зависить только от
длины вектора скорости \(R\), но не от угла \(U\). Для удобства запишем
её как функцию квадрата \(R\):

\[
f(v_x, v_y) = h(v_x^2 + v_y^2)
\]

По предпосылке М2 компоненты \(V_x\) и \(V_y\) независимы, поэтому
совместная функция плотности должна раскладываться в произведение
частных плотностей. Для удобства выразим их также через квадраты
составляющих скорости:

\[
f(v_x, v_y) = f(v_x) \cdot f(v_y) = g(v_x^2) \cdot g(v_y^2)
\]

В итоге мы получили забавное соотношение \[
h(v_x^2 + v_y^2) = g(v_x^2) \cdot g(v_y^2)
\]

Функция от суммы равна произведению функций: \[
h(a + b) = g(a)g(b)
\]

\subsection{Дифференциальное уравнение}\label{-}

\begin{itemize}
\tightlist
\item
  Как связаны \(h(a)\) и \(h'(a)\)?
\end{itemize}

Возьмём \(b=0\), получим, что \(h(a)=g(a)g(0)\).

Возьмём производную по \(b\): \[
h'(a+b)=g(a)g'(b)
\]

Подставим \(b=0\), получим \[
h'(a)=g(a)g'(0)
\]

Итого, получаем, что \[
\frac{h'(a)}{h(a)} = \frac{g'(0)}{g(0)}
\]

Другими словами производная \(h'(a)\) равна исходной функции \(h(a)\),
умноженной на константу \(k=\frac{g'(0)}{g(0)}\):

\[
h'(a) = h(a) \cdot k
\]

Этому условию удовлетворяет только функция \(h(a) = e^{ka}\) и
пропорциональные ей функции, то есть

\[
h(a) = c \cdot e^{ka}
\]

Таким образом мы нашли вид совместной функции плотности величин \(V_x\)
и \(V_y\):

\[
f(v_x, v_y) = h(v_x^2 + v_y^2) = c \cdot e^{k(v_x^2 + v_y^2)} 
\]

Величины \(V_x\) и \(V_y\) одинаково распределены, независимы, поэтому
частная функция плотности \(V_x\) имеет вид \[
f(v_x) = \sqrt{c} \cdot e^{kv_x^2}
\]

Осталось лишь найти константы \(c\) и \(k\)!

\subsection{\texorpdfstring{В поисках \(k\)}{В поисках k}}\label{--k}

Прежде всего заметим, что \(k<0\). Если бы константа \(k\) была бы
больше нуля, то тогда с ростом \(x\) экспонента \(e^{kv_x^2}\) уходила
бы на бесконечность, и площадь под функцией плотности \(f(v_x)\) не
равнялась бы единице.

Мы уже знаем, что \(\E(V_x)=0\), а единицы измерения скорости выбраны
так, что \(\Var(V_x)=1\). Замечаем, что в нашем случае
\(\Var(V_x)=\E(V_x^2)\). Осталось решить уравнение \(\E(V_x^2)=1\) и
найти \(k\).

Переходим к интегралам! Мы будем брать его по частям!

\begin{multline}
\E(V_x^2)=\int_{-\infty}^{+\infty} v_x^2 f(v_x) \, dv_x = 
  \int_{-\infty}^{+\infty} v_x \cdot v_x \cdot \sqrt{c} \cdot e^{kv_x^2} \, dv_x =\\
=\left. v_x \cdot \sqrt{c} \cdot e^{kv_x^2} \cdot (k/2) \right\rvert_{v_x = -\infty}^{v_x=+\infty} - \int_{-\infty}^{+\infty} 1 \cdot \sqrt{c} \cdot e^{kv_x^2} \cdot (1/2k) \, dv_x
\end{multline}

Замечаем, что уменьшаемое равно нулю: \[
\left. v_x \cdot \sqrt{c} \cdot e^{kv_x^2} \cdot (k/2) \right\rvert_{v_x = -\infty}^{v_x=+\infty} = 0.
\]

А вычитаемое можно записать через исходную функцию плотности:

\[
\int_{-\infty}^{+\infty} 1 \cdot \sqrt{c} \cdot e^{kv_x^2} \cdot (1/2k) \, dv_x = \frac{1}{2k} \int_{-\infty}^{+\infty}f(v_x) \, dx = \frac{1}{2k} \cdot 1
\]

Следовательно, \[
\E(V_x^2)= -\frac{1}{2k} = 1
\]

Отсюда \(k=-\frac{1}{2}\).

\subsection{\texorpdfstring{В поисках \(c\)}{В поисках c}}\label{--c}

Теперь мы знаем, что совместная функция плотности величин \(V_x\) и
\(V_y\) имеет вид \[
f(v_x, v_y) = h(v_x^2 + v_y^2) = c \cdot e^{-\frac{1}{2}(v_x^2 + v_y^2)} 
\]

\begin{itemize}
\tightlist
\item
  Какой вид имеет совместная функция плотности величин \(R\) и \(U\)?
\end{itemize}

Удобнее работать не с плотностями, а с дифференциальными формами \[
\P(V_x \in [v_x;v_x + dv_x], V_y \in [v_y;v_y + dv_y]) \sim f(v_x, v_y) dv_x \wedge dv_y
\]

Подставим \(v_x = r\cos u\) и \(v_y = r\sin u\). После упрощения
получим, что \(dv_x \wedge dv_y = r \cdot dr \wedge du\).

\[
\P(R \in [r;r+dr], U\in [u;u+du]) \sim f(r\cos u, r\sin u) r \cdot dr \wedge du = c \cdot r \cdot e^{-r^2/2} \cdot dr \wedge du
\]

Следовательно, совместная функция плотности величин \(R\) и \(U\) имеет
вид \[
f(r, u) = 
\begin{cases}
c \cdot r \cdot e^{-r^2/2}, \text{ при } r>0, u \in [0;2\pi) \\
0, \text{ иначе} \\
\end{cases}
\]

Заметим, что функция совместная функция плотности раскладывается в
произведение \(f(r, u) = f(r) \cdot f(u)\), поскольку величины \(R\) и
\(U\) независимы. Вопрос лишь в том, как поделить константу \(c\) в этом
разложении \(f(r, u)\) на сомножетели \(f(r)\) и \(f(u)\).

Интеграл \(\int r e^{-r^2/2} dr\) легко берётся: \[
\int_{-\infty}^{+\infty} r e^{-r^2/2} dr = 1
\]

Следовательно, \(f(r) = r e^{-r^2/2}\) и \(f(u) = c\).

Мы уже знаем, что величина \(U\) равномерна на \([0;2\pi]\),
следовательно, \(c=1/2\pi\).

\subsection{Нормальная стандартная величина}\label{--}

Мы пришли к выводу, что функция плотности величины \(V_x\) имеет вид \[
f(v_x) = \frac{1}{\sqrt{2\pi}} e^{-\frac{1}{2}v_x^2}
\]

При этом \(\E(V_x)=0\) и \(\Var(V_x)=1\).

Величина с такой функцией плотности называется стандартной нормальной
случайной величиной и обозначается \(N(0;1)\).

\subsection{Другие единицы измерения скорости}\label{---}

Рассмотрим линейной преобразование величины \(V_x\),
\(W = \mu + \sigma V_x\). Можно найти ожидаемое значение и дисперсию
\(W\), \(\E(W) = \mu\), \(\Var(W) = \sigma^2\).

Найдём функцию плотности \(W\). Естественнее работать не с плотностью, а
с дифференциальной формой: достаточно подставить в неё выражение для
\(v_x\), \(v_x = \frac{w-\mu}{\sigma}\).

\begin{multline}
\P(V_x \in [v_x;v_x + dv_x]) \sim f(v_x) dv_x = f\left(\frac{w-\mu}{\sigma}\right) d\frac{w-\mu}{\sigma} = \frac{1}{\sigma} f\left(\frac{w-\mu}{\sigma}\right) dw = \\
= \frac{1}{\sqrt{2\pi} \sigma} e^{-\frac{(w-\mu)^2}{2\sigma^2}} \sim \P(W \in [w;w + dw])
\end{multline}

Отсюда функция плотности величины \(W\) равна \[
f(w) =  \frac{1}{\sqrt{2\pi} \sigma} e^{-\frac{(w-\mu)^2}{2\sigma^2}}
\]

Величина с такой функцией плотностью называется нормальной случайной
величиной с математическим ожиданием \(\mu\) и дисперсией \(\sigma^2\) и
обозначается \(N(\mu; \sigma^2)\).

\printbibliography


\end{document}
