\documentclass[11pt,russian,]{article}
\usepackage{lmodern}
\usepackage{amssymb,amsmath}
\usepackage{ifxetex,ifluatex}
\usepackage{fixltx2e} % provides \textsubscript
\ifnum 0\ifxetex 1\fi\ifluatex 1\fi=0 % if pdftex
  \usepackage[T1]{fontenc}
  \usepackage[utf8]{inputenc}
\else % if luatex or xelatex
  \ifxetex
    \usepackage{mathspec}
  \else
    \usepackage{fontspec}
  \fi
  \defaultfontfeatures{Ligatures=TeX,Scale=MatchLowercase}
    \setmainfont[]{Linux Libertine O}
    \setsansfont[]{Linux Libertine O}
    \setmonofont[Mapping=tex-ansi]{Linux Libertine O}
\fi
% use upquote if available, for straight quotes in verbatim environments
\IfFileExists{upquote.sty}{\usepackage{upquote}}{}
% use microtype if available
\IfFileExists{microtype.sty}{%
\usepackage{microtype}
\UseMicrotypeSet[protrusion]{basicmath} % disable protrusion for tt fonts
}{}
\usepackage[margin=1in]{geometry}
\usepackage{hyperref}
\hypersetup{unicode=true,
            pdftitle={Дифференциальные формы вместо плотностей},
            pdfauthor={Винни-Пух},
            pdfborder={0 0 0},
            breaklinks=true}
\urlstyle{same}  % don't use monospace font for urls
\ifnum 0\ifxetex 1\fi\ifluatex 1\fi=0 % if pdftex
  \usepackage[shorthands=off,main=russian]{babel}
\else
  \usepackage{polyglossia}
  \setmainlanguage[]{russian}
\fi
\usepackage[style=alphabetic]{biblatex}

\usepackage{graphicx,grffile}
\makeatletter
\def\maxwidth{\ifdim\Gin@nat@width>\linewidth\linewidth\else\Gin@nat@width\fi}
\def\maxheight{\ifdim\Gin@nat@height>\textheight\textheight\else\Gin@nat@height\fi}
\makeatother
% Scale images if necessary, so that they will not overflow the page
% margins by default, and it is still possible to overwrite the defaults
% using explicit options in \includegraphics[width, height, ...]{}
\setkeys{Gin}{width=\maxwidth,height=\maxheight,keepaspectratio}
\IfFileExists{parskip.sty}{%
\usepackage{parskip}
}{% else
\setlength{\parindent}{0pt}
\setlength{\parskip}{6pt plus 2pt minus 1pt}
}
\setlength{\emergencystretch}{3em}  % prevent overfull lines
\providecommand{\tightlist}{%
  \setlength{\itemsep}{0pt}\setlength{\parskip}{0pt}}
\setcounter{secnumdepth}{0}
% Redefines (sub)paragraphs to behave more like sections
\ifx\paragraph\undefined\else
\let\oldparagraph\paragraph
\renewcommand{\paragraph}[1]{\oldparagraph{#1}\mbox{}}
\fi
\ifx\subparagraph\undefined\else
\let\oldsubparagraph\subparagraph
\renewcommand{\subparagraph}[1]{\oldsubparagraph{#1}\mbox{}}
\fi

%%% Use protect on footnotes to avoid problems with footnotes in titles
\let\rmarkdownfootnote\footnote%
\def\footnote{\protect\rmarkdownfootnote}

%%% Change title format to be more compact
\usepackage{titling}

% Create subtitle command for use in maketitle
\newcommand{\subtitle}[1]{
  \posttitle{
    \begin{center}\large#1\end{center}
    }
}

\setlength{\droptitle}{-2em}
  \title{Дифференциальные формы вместо плотностей}
  \pretitle{\vspace{\droptitle}\centering\huge}
  \posttitle{\par}
  \author{Винни-Пух}
  \preauthor{\centering\large\emph}
  \postauthor{\par}
  \predate{\centering\large\emph}
  \postdate{\par}
  \date{2017-10-10}

\newfontfamily{\cyrillicfonttt}{Linux Libertine O}
\newfontfamily{\cyrillicfont}{Linux Libertine O}
\newfontfamily{\cyrillicfontsf}{Linux Libertine O}

\begin{document}
\maketitle

{
\setcounter{tocdepth}{2}
\tableofcontents
}
\subsubsection{Крамольная мысль}\label{-}

Рассказать второму курсу, как преобразовывать одномерные и совместные
функции плотности с помощью дифференциальных форм. Это позволит
рассмотреть гамма и бета распределения, например.

Ни один определитель в ходе съёмок не пострадал :)

\subsubsection{Кратко про дифференциальные формы}\label{---}

Неформальное определение. Дифференциальная форма --- это объект, который
приятно интегрировать. Обозначается так:

\[
f(x, y) \, dx \wedge dy
\]

Основное свойство дифференциальных форм: \[
dx \wedge dy = - dy \wedge dx
\]

Из основного свойства немедленно следует, что \[
dx \wedge dx = - dx \wedge dx = 0
\]

\subsubsection{Поехали!}

Пример 1. Одномерная случайная величина

Вероятность попасть в отрезок \([x, x+dx]\) описывается формой с
точностью до \(o(dx)\): \[
P(X \in [x; x+dx]) \sim \begin{cases}
2x dx; \text{ если } x\in [0;1] \\
0; \text{ иначе.}
\end{cases}
\]

\begin{itemize}
\tightlist
\item
  Найдите примерно \(P(X \in [0.7, 0.71])\)
\item
  Найдите точно вероятность \(P(X \in [0.7, 0.71])\)
\item
  Найдите дифференциальную форму для \(Y=X^3\).
\item
  Найдите функцию плотности \(Y\)
\end{itemize}

Примерно:

\[
P(X\in [0.7; 0.71]) \approx 2\cdot 0.7 \cdot 0.01 = 0.014
\] Точно --- через интеграл.

Выразим \(X\) через \(Y\), \(X=Y^{1/3}\) и подставим \(x\) в
дифференциальную форму: \[
2xdx = 2(y^{1/3})d(y^{1/3}) = \frac{2}{3}y^{-1/3}dy
\]

И полностью, с учётом границ: \[
P(Y \in [y; y+dy]) \sim \begin{cases}
\frac{2}{3}y^{-1/3}dy, \text{ если } y \in [0;1] \\
0, \text{ иначе.}
\end{cases} 
\]

Функция плотности --- это то, что стоит перед \(dy\):

\[
f(y) = \begin{cases}
\frac{2}{3}y^{-1/3}, \text{ если } y \in [0;1] \\
0, \text{ иначе.}
\end{cases} 
\]

Пример 2. Переход к полярным координатам:

\[
\int \int f(x, y) \, dx \wedge dy 
\]

Хотим сделать замену \(x = r \cos \phi\), \(y = r\sin \phi\). Получаем,
что

\begin{multline}
dx \wedge dy = (\cos \phi dr - r \sin \phi d\phi) \wedge (\sin \phi dr + r\cos \phi d\phi) = \\
=(\text{неважно}) \, dr \wedge dr + (r\cos^2\phi) \, dr \wedge d\phi + (-r\sin^2\phi) d\phi \wedge dr + (\text{неважно}) d\phi \wedge d\phi = \\
= (r\cos^2\phi) \, dr \wedge d\phi + (r\sin^2\phi) dr \wedge d\phi = r dr \wedge d\phi 
\end{multline}

Отсюда получаем правило перехода к полярным медведям: \[
\int \int f(x, y) \, dx \wedge dy = \int \int f(r\cos\phi, r\sin \phi) r \, dr \wedge d \phi
\]

Пример 3. Время за которое студент Вовочка решит две задачи --- это две
независимых экспоненциально распределенных случайных величины с
параметром \(\lambda\): \(X_1\), \(X_2\). Рассмотрим новую пару величин:
суммарное время решения двух задач, \(S = X_1 + X_2\), и долю времени на
первую задачу, \(Y_1 = X_1 / S\).

\begin{itemize}
\tightlist
\item
  Найдите совместную плотность \(Y_1\) и \(S\);
\item
  Являются ли \(Y_1\) и \(S\) независимыми?
\item
  Найдите частные плотности \(Y_1\) и \(S\) с точностью до сомножителя
\end{itemize}

Поехали!

Выражаем \(X_1\) и \(X_2\) через новые переменные:
\(X_1 = Y_1 \cdot S\), \(X_2 = (1 - Y_1) S\). Находим, что
\(dx_1 = sdy_1 + y_1 ds\), \(dx_2 = (1-y_1)ds -sdy_1\). И отсюда

\begin{multline}
dx_1 \wedge dx_2 = (sdy_1 + y_1 ds) \wedge ((1-y_1)ds -sdy_1) = \\
= s(1-y_1) \; dy_1 \wedge ds -y_1 s \; ds \wedge dy_1 + (\text{неважно}) \; ds\wedge ds + (\text{неважно}) \; dy_1 \wedge dy_1 = \\
= s(1-y_1) \; dy_1 \wedge ds + y_1 s \; dy_1 \wedge ds = s dy_1 \wedge ds
\end{multline}

Упрощаем дифференциальную форму в целом: \[
f(x_1, x_2) dx_1 \wedge dx_2 = \lambda \exp(-\lambda x_1) \lambda \exp(-\lambda x_2) \, dx_1 \wedge dx_2 = \lambda^2 \exp( -\lambda s) s dy_1 \wedge ds
\]

Отсюда новая совместная плотность равна:

\[
f(y_1, s) = \begin{cases}
\lambda^2 \exp( -\lambda s) s, \text{ если } y_1 \in [0;1], s \geq 0 \\
0, \text{ иначе.}
\end{cases}
\]

Поскольку функция плотности распадается в произведение плотностей, можно
сделать вывод, что величины \(S = X_1 + X_2\) и \(Y_1 = X_1/S\)
независимы. Частные функции плотности с точностью до константы можно
получить без интегрирования:

\[
f(y_1) \propto 1 \; \text{ если } y_1 \in [0;1] 
\] Получается, что в данном случае константа пропорциональности равна 1,
то есть \[
f(y_1) = 1 \; \text{ если } y_1 \in [0;1] 
\]

И, следовательно, \[
f(s) = \lambda^2 \exp( -\lambda s) s \; s \geq 0
\]

\subsubsection{Задачки для самостоятельного решения}\label{---}

\paragraph{Занудная задача}\label{-}

Совместная функция плотности величин \(X\) и \(Y\) имеет вид \[
f(x, y) = \begin{cases}
4xy, \text{ если } x, y \in [0;1];\\
0, \text{ иначе }
\end{cases}
\]

\begin{itemize}
\tightlist
\item
  Выпишите дифференциальную форму вероятностей для пары \(X\) и \(Y\)
\item
  Найдите дифференциальную форму для \(S = X+Y\) и \(R = X/(X + Y)\).
\item
  Найдите совместную функцию плотности для \(S = X+Y\) и
  \(R = X/(X + Y)\).
\end{itemize}

\paragraph{Преобразование Бокса-Мюллера}\label{--}

Величины \(U_1\) и \(U_2\) независимы и равномерны \(U[0;1]\).
Рассмотрим пару величин \(Y_1 = R\cdot \cos \alpha\),
\(Y_2 = R\cdot \sin \alpha\), где \(R=\sqrt{-2\ln U_1}\), а
\(\alpha = 2\pi U_2\).

\begin{itemize}
\tightlist
\item
  Выпишите дифференциальную форму для пары \(U_1\), \(U_2\)
\item
  Выпишите дифференциальную форму для пары \(Y_1\), \(Y_2\)
\item
  Найдите совместный закон распределения \(Y_1\) и \(Y_2\);
\item
  Верно ли, что \(Y_1\) и \(Y_2\) независимы?
\item
  Как распределены \(Y_1\) и \(Y_2\) по отдельности?
\end{itemize}

\paragraph{Вывод плотности бета-распределения}\label{---}

В Пуассоновском потоке с интенсивностью \(\lambda\) время наступления
\(k\)-го события распределено согласно гамма-распределению
\(Gamma(k, \lambda)\) и имеет вероятностную дифференциальную форму

\[
P(Y \in [y; y+dy]) = const \cdot y^{k-1} \cdot \exp(-\lambda y), \text{ при } y\geq 0;
\]

До прихода Ёжика Медвежонок сидел на крылечке один и насчитал \(k_1\)
падающую звезду. А после прихода Ёжика Медвежонок насчитал ещё \(k_2\)
падающих звёзд. Пусть \(Y_1\) и \(Y_2\) --- время, которое которое
наблюдал за звёздами Медвежонок до и после прихода Ёжика.

\begin{itemize}
\tightlist
\item
  Выпишите совместную дифференциальную форму для \(Y_1\) и \(Y_2\)
\item
  Найдите совместную дифференциальную форму для \(S=Y_1 + Y_2\) и
  \(Z=Y_1/S\).
\item
  С точностью до сомножителя выпишите дифференциальную форму для доли
  времени, в течение которого Медвежонок наблюдал звёзды один.
\item
  Выпишите функцию плотности бета-распределения
\end{itemize}

\subsubsection{Источники мудрости}\label{-}

\begin{itemize}
\tightlist
\item
  Всё началось с одного ответа на форуме,
  \href{https://stats.stackexchange.com/questions/36093}{whuber, there's
  better way}
\item
  \href{https://arxiv.org/abs/math/0306194v1}{David Bachman} Geometric
  approach to differential forms. Черновая версия опубликованной книжки.
\item
  \href{https://arxiv.org/abs/1709.08492}{Jonathan Gratus} A pictorial
  introduction to differential geometry. Ни одной формулы и только куча
  цветных картинок!
\item
  \href{https://arxiv.org/abs/1604.07862}{Lorenzo Sadun} Lecture Notes
  on Differential Forms. Приятные лекции.
\end{itemize}

\printbibliography


\end{document}
