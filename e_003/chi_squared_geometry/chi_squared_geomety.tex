\documentclass[11pt,russian,]{article}
\usepackage{lmodern}
\usepackage{amssymb,amsmath}
\usepackage{ifxetex,ifluatex}
\usepackage{fixltx2e} % provides \textsubscript
\ifnum 0\ifxetex 1\fi\ifluatex 1\fi=0 % if pdftex
  \usepackage[T1]{fontenc}
  \usepackage[utf8]{inputenc}
\else % if luatex or xelatex
  \ifxetex
    \usepackage{mathspec}
  \else
    \usepackage{fontspec}
  \fi
  \defaultfontfeatures{Ligatures=TeX,Scale=MatchLowercase}
    \setmainfont[]{Linux Libertine O}
    \setsansfont[]{Linux Libertine O}
    \setmonofont[Mapping=tex-ansi]{Linux Libertine O}
\fi
% use upquote if available, for straight quotes in verbatim environments
\IfFileExists{upquote.sty}{\usepackage{upquote}}{}
% use microtype if available
\IfFileExists{microtype.sty}{%
\usepackage{microtype}
\UseMicrotypeSet[protrusion]{basicmath} % disable protrusion for tt fonts
}{}
\usepackage[margin=1in]{geometry}
\usepackage{hyperref}
\hypersetup{unicode=true,
            pdftitle={Геометрия хи-квадрат распределения},
            pdfauthor={Винни-Пух},
            pdfborder={0 0 0},
            breaklinks=true}
\urlstyle{same}  % don't use monospace font for urls
\ifnum 0\ifxetex 1\fi\ifluatex 1\fi=0 % if pdftex
  \usepackage[shorthands=off,main=russian]{babel}
\else
  \usepackage{polyglossia}
  \setmainlanguage[]{russian}
\fi
\usepackage[style=alphabetic]{biblatex}

\addbibresource{chi-squared.bib}
\usepackage{graphicx,grffile}
\makeatletter
\def\maxwidth{\ifdim\Gin@nat@width>\linewidth\linewidth\else\Gin@nat@width\fi}
\def\maxheight{\ifdim\Gin@nat@height>\textheight\textheight\else\Gin@nat@height\fi}
\makeatother
% Scale images if necessary, so that they will not overflow the page
% margins by default, and it is still possible to overwrite the defaults
% using explicit options in \includegraphics[width, height, ...]{}
\setkeys{Gin}{width=\maxwidth,height=\maxheight,keepaspectratio}
\IfFileExists{parskip.sty}{%
\usepackage{parskip}
}{% else
\setlength{\parindent}{0pt}
\setlength{\parskip}{6pt plus 2pt minus 1pt}
}
\setlength{\emergencystretch}{3em}  % prevent overfull lines
\providecommand{\tightlist}{%
  \setlength{\itemsep}{0pt}\setlength{\parskip}{0pt}}
\setcounter{secnumdepth}{0}
% Redefines (sub)paragraphs to behave more like sections
\ifx\paragraph\undefined\else
\let\oldparagraph\paragraph
\renewcommand{\paragraph}[1]{\oldparagraph{#1}\mbox{}}
\fi
\ifx\subparagraph\undefined\else
\let\oldsubparagraph\subparagraph
\renewcommand{\subparagraph}[1]{\oldsubparagraph{#1}\mbox{}}
\fi

%%% Use protect on footnotes to avoid problems with footnotes in titles
\let\rmarkdownfootnote\footnote%
\def\footnote{\protect\rmarkdownfootnote}

%%% Change title format to be more compact
\usepackage{titling}

% Create subtitle command for use in maketitle
\newcommand{\subtitle}[1]{
  \posttitle{
    \begin{center}\large#1\end{center}
    }
}

\setlength{\droptitle}{-2em}
  \title{Геометрия хи-квадрат распределения}
  \pretitle{\vspace{\droptitle}\centering\huge}
  \posttitle{\par}
  \author{Винни-Пух}
  \preauthor{\centering\large\emph}
  \postauthor{\par}
  \predate{\centering\large\emph}
  \postdate{\par}
  \date{2018-01-10}

\usepackage{bbm}
\newfontfamily{\cyrillicfonttt}{Linux Libertine O}
\newfontfamily{\cyrillicfont}{Linux Libertine O}
\newfontfamily{\cyrillicfontsf}{Linux Libertine O}
\newcommand{\cN}{\mathcal{N}}
\newcommand{\E}{\mathbb{E}}
\newcommand{\RR}{\mathbb{R}}
\renewcommand{\Rn}{\RR^n}
\newcommand{\1}{\mathbbm{1}}
\newcommand{\Lin}{\mathcal{L}in}
\newcommand{\Linp}{\Lin^{\perp}}
\newcommand{\col}{\mathcal{col}}
\newcommand{\colp}{\col^{\perp}}
\newcommand{\sVar}{s\mathbb{V}ar}
\newcommand{\Var}{\mathbb{V}ar}
\newcommand{\Cov}{\mathbb{C}ov}
\renewcommand{\P}{\mathbb{P}}
\renewcommand{\col}{\mathrm{col}}
\newcommand{\trace}{\mathrm{trace}}
\newcommand{\rang}{\mathrm{rang}}

\begin{document}
\maketitle

{
\setcounter{tocdepth}{2}
\tableofcontents
}
\subsection{Пространства и подпространства}\label{--}

Если говорить совсем просто, то пространство \(\Rn\) --- это все
столбики, состоящие из \(n\) действительных чисел. А если вспоминать
определение, то линейное пространство --- это такой набор векторов, в
котором:

\begin{enumerate}
\def\labelenumi{\arabic{enumi}.}
\tightlist
\item
  Разрешено складывать два любых вектора и результат остаётся внутри
  набора;
\item
  Разрешено умножать любой вектор на любое число и результат остаётся
  внитри набора;
\item
  Сложение векторов и умножение вектора на число согласованы между
  собой.
\end{enumerate}

Подпространство --- это часть набора, которая сама по себе является
пространством. Есть два популярных способа описать подпространство
внутри \(\Rn\):

\begin{enumerate}
\def\labelenumi{\arabic{enumi}.}
\tightlist
\item
  Линейная оболочка некоторого набора векторов.
\end{enumerate}

Например, внутри \(\RR^{3}\) есть два вектора \(x = (1, 1, 1)\) и
\(y = (1, 0, 0)\) и подпространство \(V\), образованное ими \[
V = \Lin (x, y),
\] то есть это все вектора вида \(\alpha x + \beta y\), где \(\alpha\) и
\(\beta\) --- произвольные числа.

\begin{enumerate}
\def\labelenumi{\arabic{enumi}.}
\setcounter{enumi}{1}
\tightlist
\item
  Ортогональное дополнение к набору векторов.
\end{enumerate}

Например, внутри \(\RR^{3}\) есть два вектора \(x = (1, 1, 1)\) и
\(y = (1, 0, 0)\) и подпространство \(W\) всех векторов,
перпендикулярным им обоим \[
W = \Linp (x, y).
\]

Напомним, два вектора \(a\) и \(b\) в пространстве \(\Rn\)
перпендикулярны, если их скалярное произведение \(\langle a, b \rangle\)
равно нулю. Поэтому подпространство \(W\) можно также описать системой
уравнений.

Подпространство \(W\) состоит из всех векторов \(w=(w_1, w_2, w_3)\),
удовлетворяющих системе:

\[
\begin{cases}
w_1 \cdot 1 + w_2 \cdot 1 + w_3 \cdot 1 = 0 \\
w_1 \cdot 1 + w_2 \cdot 0 + w_3 \cdot 0 = 0 \\
\end{cases}
\]

Если вектора \(a_1\), \ldots, \(a_k\) линейно независимы и лежат внутри
\(\Rn\), то размерности описанных нами подпространств равны

\[
\dim \Lin(a_1, \ldots, a_k) = k
\]

\[
\dim \Linp(a_1, \ldots, a_k) =  n - k
\]

\paragraph{Упражнения}

\begin{enumerate}
\def\labelenumi{\arabic{enumi}.}
\item
  Лежит ли вектор \(a=(1,2,3)\) в подпространстве
  \(\Lin((1,1,1), (0, 2, 4))\)?
\item
  Найдите базис в ортогональном дополнении подпространства
  \(\Lin((1, 2, 3))\).
\item
  Найдите размерность пространств \(\Lin((1, 2, 3, 4))\),
  \(\Lin((1, 2, 3), (1, 2, 1))\), \(\Linp((1, 2, 3, 4, 5))\).
\end{enumerate}

\subsection{Проекции}

Проекцией вектора \(a\) на подпространство \(L\) называется вектор
\(\hat a\), лежащий в \(L\) и ближайший к \(a\).

Есть два популярных способа найти проекцию:

\begin{enumerate}
\def\labelenumi{\arabic{enumi}.}
\item
  Решить задачу минимизации расстояния \[
  \min_{\hat a\in L} ||a - \hat a||
  \]
\item
  Потребовать, чтобы разница \(a-\hat a\) была перпендикулярна любому
  вектору из \(L\): \[
  a- \hat a \in L^{\perp}
  \]
\end{enumerate}

\paragraph{Упражнения:}\label{-1}

\begin{itemize}
\item
  Спроецируйте вектор \(z=(1, 2, 5, 4)\) на подпространство, порождённое
  вектором \((1, 1, 1, 1)\). Найдите квадрат длины проекции.
\item
  Спроецируйте произвольный вектор \(z\) на прямую, порождённую вектором
  \$b = (1, 1, 1, 1) \$. Найдите косинус угла между исходным вектором и
  проекцией.
\item
  Спроецируйте вектор \(z=(1, 2, 5, 4, 3)\) на ортогональное дополнение
  к вектору \((1, 1, 1, 1, 1)\).
\item
  Спроецируйте вектор \(z=(1, 1, 2, 2, 2)\) на вектор единичной длины
  \(v = (0.5, 0.5, 0.5, 0.3, 0.4)\).
\item
  Спроецируете вектор \(z=(1, 1, 2, 2, 2)\) на пространство
  \(\Linp(v)\), где \(v = (0.5, 0.5, 0.5, 0.3, 0.4)\) --- вектор
  единичной длины. Найдите квадрат длины проекции.
\item
  Спроецируйте произвольный вектор \(z\) на произвольный вектор \(v\)
  единичной длины.
\end{itemize}

\subsection{Хи-квадрат распределение}\label{--}

Определение. Пусть компоненты \(n\)-мерного вектора \(z\) имеют
стандартное нормальное распределение, \(z_i \sim \cN(0;1)\) и
независимы. Рассмотрим произвольное фиксированное \(k\)-мерное
подпространство \(L\). Абсолютно любое. Обозначим проекцию вектора \(z\)
на подпространство \(L\) буквой \(\hat z\), а квадрат длины проекции ---
буквой \(Q\):

\[
Q = ||\hat z||^2 = \langle \hat z, \hat z\rangle = \hat z'\hat z
\]

Закон распределения случайной величины \(Q\) называется хи-квадрат
распределением с \(k\)-степенями свободы.

Пример. Вектор \(z \in R^3\), компоненты \(z_i \sim \cN(0;1)\) и
независимы. Найдите явную формулу для величины \(Q\), квадрата длины
проекции \(z\) на плоскость \(z_1 + z_2 + z_3 =0\). Какое распределение
имеет \(Q\)?

Пример. Вектор \(z \in R^3\), компоненты \(z_i \sim \cN(0;1)\) и
независимы. Найдите явную формулу для величины \(Q\), квадрата длины
проекции \(z\) на прямую, порожденную вектором \(a = (1, 1, 1)\). Какое
распределение имеет \(Q\)?

Пример. Вектор \(z \in R^7\), компоненты \(z_i \sim \cN(0;1)\) и
независимы. Какое распределение имеет величина \(Q\), квадрат длины
проекции \(z\) на подпространство, задаваемое системой уравнений \[
\begin{cases}
z_1 + z_2 + z_3 +z_4 + z_5 +z_6 + z_7 = 0 \\
z_1 + 2z_2 +3z_3 +4z_4 +5z_5+6z_6+7z_7 =0 \\
\end{cases}
\]

Пример. Вектор \(z \in R^4\), компоненты \(z_i \sim \cN(7;1)\) и
независимы. Какое распределение имеет величина \(Q\), квадрат длины
проекции \(z\) на подпространство, ортогональное прямой, порождаемой
вектором \(a=(1, 1, 1, 1)\)?

\subsection{Связь со стандартным определением}\label{---}

Если взять практически любой учебник, то там будет дано другое
определение \(\chi^2\)-распределения.

Величина \(Q\) имеет \(\chi^2\)-распределение с \(k\) степенями свободы,
если она представима в виде \[
Q = z_1^2 + z_2^2 + \ldots + z_k^2,
\] где \(z_i\) независимы и стандартны нормальны, \(z_i \sim \cN(0;1)\).

Сначала заметим, что стандартное определение из учебника --- частный
случай нашего. Что получится если вектор
\(z = (z_1, z_2, \ldots, z_n)\), лежащий в \(\Rn\), спроецировать на
\(k\)-мерное подпространство \(V\) всех векторов, у которых первые \(k\)
координат произвольные, а остальные --- нули?

Получится вектор \(\hat z = (z_1, z_2, \ldots, z_k, 0, 0, \ldots, 0)\).
И квадрат длины проекции будет равен \[
Q = ||\hat z||^2 = z_1^2 + z_2^2 + \ldots + z_k^2.
\]

А наше новое определение допускает проецирование на любое \(k\)-мерное
подпространство :)

Возникает естественный вопрос, а вдруг, если спроецировать на какое-то
хитрое подпространство, скажем \(\Lin(a, b, c) \cap \Linp (d, e, f)\),
эквивалентность определений нарушится?

Вдруг возможно, что квадрат длины проекции вектора \(z\) на
подпространство размерности \(k\) не будет представляться в виде суммы
\(k\) независимых стандартных нормальных величин?

Оказывается два определения полностью эквивалентны в силу двух фактов:

\begin{enumerate}
\def\labelenumi{\arabic{enumi}.}
\item
  Закон распределения вектора \(z\) не изменится, если вектор \(z\)
  повернуть в любом направлении на произвольный угол;
\item
  Любое \(k\)-мерное подпространство всегда можно повернуть так, чтобы
  оно совпало с подпространством \(V\) всех векторов, у которых первые
  \(k\) координат произвольные, а остальные --- нули.
\end{enumerate}

Идеи доказательства:

\begin{enumerate}
\def\labelenumi{\arabic{enumi}.}
\tightlist
\item
  Функция плотности \(z\) имеет вид: \[
  f(z_1, z_2, \ldots, z_n) = f(z_1) \cdot f(z_2)\cdot \ldots \cdot f(z_n) \propto e^{-z_1^2/2}e^{-z_2^2/2}\ldots e^{-z_1^2/2} = e^{-\frac{1}{2}(z_1^2 + z_2^2 +\ldots +z_n^2)};
  \]
\end{enumerate}

Мы видим, что значение функции плотности в произвольной точке \(z\)
зависит только от расстояния от \(z\) до нуля, но не от угла.

\begin{enumerate}
\def\labelenumi{\arabic{enumi}.}
\setcounter{enumi}{1}
\tightlist
\item
  Рассмотрим стандартный базис \(e_1\), \(e_2\), \ldots, \(e_n\) в
  \(\Rn\). Рассмотрим \(k\)-мерное подпространство \(V \subset \Rn\).
  Выберем в подпространстве \(V\) произвольный ортогональный базис из
  \(k\) векторов: \(v_1\), \ldots, \(v_k\). Сначала повернём \(V\) так,
  чтобы \(v_1\) совпал с \(e_1\). Затем будем поворачивать так, чтобы
  \(v_1\) не трогать, а \(v_2\) повернуть до совпадения с \(e_2\). И так
  далее.
\end{enumerate}

\subsection{Определение в матрицах}\label{--}

Зафиксируем \(k\) линейно-независимых векторов \(x_1\), \(x_2\), \ldots,
\(x_k\). Для удобства занесём их столбцами в матрицу \(X\). То есть
\(x_j\) --- это \(j\)-ый столбец матрицы \(X\). Введём два обозначения.

Линейная оболочка всех столбцов матрицы \(X\): \[
\col X = \Lin(x_1, x_2, \ldots, x_k)
\]

Ортогональное дополнение всех столбцов матрицы \(X\): \[
\colp X = \Linp(x_1, x_2, \ldots, x_k)
\]

Проецирование --- это линейное преобразование векторов:

\begin{enumerate}
\def\labelenumi{\arabic{enumi}.}
\item
  Если вектор растянуть в \(\alpha\) раз, то проекция растянется в
  \(\alpha\) раз;
\item
  Проекция суммы двух векторов равна сумме проекций каждого вектора по
  отдельности.
\end{enumerate}

Поэтому проецирование вектора \(z\) на пространство \(\col X\) можно
записать в виде его умножения на некую матрицу \(H\):

\[
\hat z = H \cdot z
\]

Мы называем матрицу \(H\) матрицей-шляпницей (hat-matrix), потому что
она навешивает шляпку на \(z\).

Естественно матрица \(H\) зависит от того подпространства \(\col X\) на
которое мы проецируем. Осталось найти эту связь. Заметим, что вектор
\(z - \hat z\) перпендикулярен пространству \(\col X\). То есть \[
z - \hat z \perp X
\]

Столбцы \(X\) перпендикулярны вектору \(\hat z\), только если склалярное
произведение \(\hat z\) с каждым столбцом \(X\) равно нулю: \[
X' \cdot (z - \hat z) = 0
\]

Вектор \(\hat z\) лежит в подпространстве \(\col X\), поэтому он должен
выражаться через столбцы матрицы \(X\):

\[
\hat z = X\cdot \alpha
\]

Получаем уравнение на веса \(\alpha\):

\[
X' (z - X\alpha)=0
\]

После раскрытия скобок имеем:

\[
X'X\alpha = X'z
\]

Временно предположим, что матрица \(X'X\) обратима: \[
\alpha = (X'X)^{-1}X'z
\]

И наконец,

\[
Hz = \hat z = X\alpha = X(X'X)^{-1}X'z
\]

Таким образом, проецирование на линейное подпространство \(\col X\)
можно задать в виде умножения на матрицу \[
H=X(X'X)^{-1}X'
\]

У матрицы-шляпницы \(H\) много приятных свойств. Например, необходимое и
достаточное условие, чтобы некая матрица \(H\) задавала проецирование:

\[
\begin{cases}
H' = H \\
H^2 = H \\
\end{cases}
\]

Геометрическая интерпретация:

\begin{enumerate}
\def\labelenumi{\arabic{enumi}.}
\tightlist
\item
  \(H'=H\). Для любых двух векторов \(x\) и \(y\) скалярное произведение
  спроецированного \(x\) на исходный \(y\) равно скалярному произведение
  исходного \(x\) на спроецированный \(y\).
\end{enumerate}

\[
\langle Hx, y\rangle = (Hx)'y = x' H' y = x' (H'y) = x' (Hy) = \langle x, Hy \rangle
\]

\begin{enumerate}
\def\labelenumi{\arabic{enumi}.}
\setcounter{enumi}{1}
\tightlist
\item
  \(H^2=H\). Проецирование два раза эквивалентно проецированию один раз.
\end{enumerate}

Другое необходимое и достаточное условие:

\[
\begin{cases}
H' = H \\
\text{Все собственные числа } H \text{ равны } 0 \text{ или } 1\\
\end{cases}
\]

Если спроецировать нормальный стандартный вектор \(z\) на \(\col X\), то
мы получим вектор \(\hat z = Hz\). И квадрат длины \(\hat z\) будет
равен \[
||\hat z ||^2 = (Hz)'Hz=z'H'Hz = z'Hz
\]

Поэтому можно дать определение:

Величина \(Q\) имеет хи-квадрат распределение с \(k\) степенями свободы,
если она представима в виде \[
Q = z'Hz,
\] где \(z\) --- нормальный стандартный вектор, а \(H\) --- матрица,
проецирующая на \(k\)-мерное подпространство, то есть \(H'=H\),
\(H^2=H\), \(\trace H =\rang H= k\).

По сути это определение просто переводит на язык матриц идею
проецирования. Не стоит бояться матриц! Весь смысл матриц в том, чтобы
записать формально геометрическую идею!

У матрицы-шляпницы ранг и след совпадают, так как ранг --- это
количество ненулевых собственных чисел, след --- это сумма собственных
чисел, а собственные числа --- нули или единицы.

С матрицами снова можно доказать эквивалентность нового определения и
традиционного:

\begin{enumerate}
\def\labelenumi{\arabic{enumi}.}
\item
  У матрицы \(H\) собственные числа равны 0 или 1, так как \(H^2=H\).
\item
  Матрица \(H\) симметричная и допускает разложение \(H = PDP'\), где
  \(P\) --- матрица из собственных векторов единичной длины, а \(D\) ---
  диагональная матрица из собственных чисел. На диагонали \(D\) стоят
  нули и единицы, пусть для определенности единицы идут сначала, а нули
  потом.
\item
  Случайная величина \(Q\) допускает разложение: \[
  Q=z'Hz=z'PDP'z=(Pz)'\cdot D\cdot (Pz) = a'Da=\sum_{i=1}^k a_i^2
  \]
\item
  Замечаем, что компоненты вектора \(a=Pz\) имеют:
\end{enumerate}

\begin{itemize}
\tightlist
\item
  нулевое математическое ожидание: у компонент вектора \(z\) нулевое
  ожидание;
\item
  единичную дисперсию: каждая величина \(a_j\) равен столбцу матрицы
  \(P\) домноженному на вектор \(z\), а столбец матрицы \(P\) имеет
  единичную длину;
\item
  независимы: столбцы матрицы \(P\) ортогональны, поэтому дают нулевую
  ковариацию между \(a_j\) и \(a_k\);
\end{itemize}

Проецирование на линейное пространство \(\colp X\) можно задать в виде
умножения на матрицу \(M = I - H\). Поэтому квадрат длины проекции
стандартного нормального вектора \(z\) на подпространство \(\colp X\)
записывается как \[
S = ||Mz||^2 = (Mz)'Mz=z'M'Mz=z'(I-H)z
\]

И, конечно, величина \(S\) имеет хи-квадрат распределение с \(n-k\)
степенями свободы.

\subsection{Выборочная дисперсия --- геометрия}\label{--}

Начнём с упражнения. Пусть \(z\) --- вектор из \(\Rn\), а \(\1\) ---
вектор из единиц. Чему равен квадрат длины проекции \(z\) на
\(\Linp(\1)\)? Чему равна проекция вектора \(\1\) на \(\Linp(\1)\)?

Сначала спроецируем вектор \(z\) на \(\Lin(\1)\). Получаем вектор
\(\bar z \cdot \1 = (\bar z, \bar z, \ldots, \bar z)\). Поэтому проекция
\(z\) на \(\Linp(\1)\) равна
\(z - \bar z \cdot \1 = (z_1 - \bar z, z_2 - \bar z, \ldots, z_n - \bar z)\).

Вектор из единиц ортогонален пространству \(\Linp(\1)\), поэтому вектор
\(\1\) проецируется в нулевой вектор.

Поэтому для стандартного нормального вектора \(z\) величина
\(\sum (z_i - \bar z)^2\) имеет хи-квадрат распределение с \(n-1\)-ой
степенью свободы.

А теперь замечаем, что выборочная дисперсия вектора \(x\) --- это
квадрат длины проекции делённый на размерность подпространства!

\[
\sVar(x) = \frac{\sum_{i=1}^n (x_i - \bar x)^2}{n-1} = \frac{1}{n-1}\cdot ||x - \bar x \cdot \1||^2
\]

Осталось добавить предположения:

Пусть \(x_i\) независимы и одинаково распределены
\(\cN(\mu, \sigma^2)\). Заметим, что вектор \(x\) можно представить в
виде \[
x = \mu \cdot 1 + \sigma z,
\] где \(z\) --- стандартный нормальный вектор.

Проекция \(x\) на \(\Linp(\1)\) совпадает с проекцией \(\sigma z\) на
\(\Linp(\1)\). Вектор \(\1\) проецируется в нулевой. А проекция
\(\sigma z\) в \(\sigma\) раз длиннее, чем проекция \(z\). Поэтому: \[
\sum (x_i - \bar x)^2 = \sigma^2 \sum_{i=1}^n (z_i - \bar z)^2
\]

Таким образом, мы доказали, что \[
\frac{\sum (x_i - \bar x)^2}{\sigma^2} = \frac{(n-1)\sVar(x)}{\sigma^2} \sim \chi^2_{n-1}
\]

\subsection{Ковариационная матрица спроецированного вектора}\label{---}

Тут хорошо бы максимально просто доказать, что если \(\hat z = Hz\), и
\(z\) --- стандартный нормальный вектор, то \(\Var(\hat z) = H\).

Ковариационная матрица вектор \(y\) определяется как \[
\Var(y) = \E[(y-\mu)(y-\mu)'],
\] где \(\mu = \E(y)\).

Эквивалентно дисперсию можно определить как \[
\Var(y) = \E(yy')- \E(y)\E(y)'
\]

Посмотрим, чему равна \(\Var(Ay)\):

\[
\Var(Ay) = \E((Ay)(Ay)') - \E(Ay)\E(Ay)' = \E(Ayy'A') - A\E(y)(A\E(y))'=A\E(yy')A' - A\E(y)\E(y)'A'=A\Var(y)A'
\]

В силу этого мы находим ещё одно шикарное свойство матрицы-шляпницы!
Пусть \(z\) --- стандартный нормальные вектор, \(z \sim \cN(0; I)\). В
частности, \(\Var(z) = I\).

Найдём ковариационную матрицу проекции \(\hat z\):

\[
\Var(\hat z) = \Var(Hz)=H\Var(z)H'=H\cdot I\cdot H'=HH'=H^2=H
\]

Матрица-шляпница \(H\) является ковариационной матрицей спроецированного
вектора!

\subsection{Хи-квадрат тест Пирсона, геометрия}\label{----}

Для начала спроецируем стандартный нормальный вектор \(z\) на
\(\Linp(v)\), где \(v\) --- единичный вектор. При этом мы получим вектор
\(\hat z = (I - H) \cdot z\):

\[
\hat z = (I - vv')\cdot z
\]

По нашем определению квадрат длины \(\hat z\) имеет хи-квадрат
распределение со степенями свободы равными размерности подпространства
\(\Linp(v)\). А размерность пространства \(\Linp(v)\) на единицу меньше
размерности исходного пространства.

Ковариационная матрица вектора \(\hat z\) имеет именно такой же вид: \[
\Var(\hat z) = (I - vv')
\]

Запомним эту ковариационную матрицу! И запомним, что она возникает у
проекции на ортогональное дополнение к вектору \(v\)! А теперь к
покемонам!

Каждый отловленный покемон может быть одного из \(r\) видов. Виды
покемонов встречаются с вероятностью \(p_1\), \ldots, \(p_{r}\). Всего
мы ловим \(n\) покемонов, \(\nu_j\) --- количество покемонов вида \(j\).

Замечаем, что \(\nu_j\) имеет биномиальное распределение
\(Bin(n, p_j)\). В частности, \(\E(\nu_j) = np_j\) и
\(\Var(\nu_j)=n p_j (1- p_j)\). Также можно установить, что \[
\Cov(\nu_j, \nu_i) = -np_ip_j
\]

Мы немного необычным образом отнормируем эти \(\nu_j\): вычтем
математическое ожидание и поделим на корень из математического ожидания!

\[
\nu_j^* = \frac{\nu_j - np_j}{\sqrt{np_j}}
\]

При этом окажется, что: \(\E(\nu_j^*) = 0\),
\(\Var(\nu_j^*) = 1 - p_j\),
\(\Cov(\nu_i, \nu_i) = - \sqrt{p_i}\sqrt{p_j}\).

Заметим, что по центральное предельной теореме
\(\nu_j^* \to \cN(0; 1 - p_j)\).

Присмотримся повнимательнее!

\[
\Var(\nu^*) = \begin{pmatrix}
1 - p_1 & -\sqrt{p_1}\sqrt{p_2} & -\sqrt{p_1}\sqrt{p_3} & \ldots \\
-\sqrt{p_2}\sqrt{p_1} & 1 - p_2 & -\sqrt{p_2}\sqrt{p_3} & \ldots \\
-\sqrt{p_3}\sqrt{p_1} & -\sqrt{p_3}\sqrt{p_2} & 1 - p_3  & \ldots \\
\vdots & \vdots & \vdots &  \\
\end{pmatrix} = 
\begin{pmatrix}
1 & 0 & 0 & \ldots \\
0 & 1 & 0 & \ldots \\
0 & 0 & 1 & \ldots \\
\vdots & \vdots & \vdots &  \\
\end{pmatrix} - \begin{pmatrix}
\sqrt{p_1}\sqrt{p_1} & \sqrt{p_1}\sqrt{p_2} & \sqrt{p_1}\sqrt{p_3} & \ldots \\
\sqrt{p_2}\sqrt{p_1} & \sqrt{p_2}\sqrt{p_2} & \sqrt{p_2}\sqrt{p_3} & \ldots \\
\sqrt{p_3}\sqrt{p_1} & \sqrt{p_3}\sqrt{p_2} & \sqrt{p_3}\sqrt{p_3}  & \ldots \\
\vdots & \vdots & \vdots &  \\
\end{pmatrix}
\]

Выходит, что ковариационная матрица нового вектора \(\nu^*\) представима
в виде:

\[
\Var(\nu^*) = I - vv',
\] где вектор \(v\) состоит из корней вероятностей,
\(v= (\sqrt{p_1}, \sqrt{p_2}, \ldots, \sqrt{p_r})\). Заметим, что вектор
\(v\) имеет единичную длину:

\[
||v||^2 = v_1^2 + v_2^2 + \ldots + v_r^2 = p_1 + p_2 + \ldots + p_r = 1
\]

То есть ковариационная матрица вектора \(\nu^*\) совпадает с
ковариационной матрицой проекции вектора \(z\) из \(\RR^r\) на
подпространство \(\Linp(v)\). Закон распределения многомерного
нормального вектора однозначно определяется вектором математических
ожиданий и ковариационной матрицей.

Следовательно, сумма \(\sum_{j=1}^r (\nu_j^*)^2\) распределена при
больших \(n\) также, как квадрат длины проекции \(z\) на \(\Linp(v)\).

Поэтому \[
\sum_{j=1}^r (\nu_j^*)^2 = \sum_{j=1}^r \frac{(\nu_j - np_j)^2}{np_j} \to \chi^2_{r-1}
\]

Недостатки доказательства:

\begin{enumerate}
\def\labelenumi{\arabic{enumi}.}
\item
  Строго говоря, ЦПТ гарантирует, что каждый \(\nu_j^*\) в отдельности
  имеет асимптотически нормальное распределение, а здесь требуется
  асимптотическая нормальность вектора \(\nu^*\), то есть требуется ЦПТ
  в векторной форме.
\item
  Деление на корень из математического ожидания выглядит магией, которая
  потом раскрывается, а хотелось бы раскрыть её по ходу.
\end{enumerate}

Аналогичное доказательство можно найти в курсе
\textcite{panchenko2005statistics}.

\subsection{Выборочная дисперсия --- явно скалярно}\label{---}

Мы помним, что \(\sum (z_i - \bar z)^2\) --- это квадрат длины проекции
вектора \(z\) на подпространство \(\Linp (\1)\). Сама проекция вектора
\(z\) на \(\Linp (\1)\) на примере \(z \in \RR^{5}\) имеет вид:

\[
\begin{pmatrix}
z_1 - \bar z \\
z_2 - \bar z \\
z_3 - \bar z \\
z_4 - \bar z \\
z_5 - \bar z \\
\end{pmatrix} 
\]

Мы легко можем выбрать ортогональный базис в подпространстве
\(\Linp (\1)\) явно. Явный базис:

\[
\Linp (\1) = \col \begin{pmatrix}
1  & 1  & 1  & 1  \\ 
-1 & 1  & 1  & 1  \\
0  & -2 & 1  & 1  \\
0  & 0  & -3 & 1  \\
0  & 0  & 0  & -4  \\
\end{pmatrix}
\]

Действительно, давайте проверим:

\begin{enumerate}
\def\labelenumi{\arabic{enumi}.}
\item
  Каждый столбец ортогонален вектору \(\1 = (1, 1, 1, 1, 1)\).
\item
  Столбцы ортогональны между собой.
\item
  Столбцов четыре :)
\end{enumerate}

Поэтому столбцы задают базис в подпространстве \(\Linp (\1)\).

Нам нужно спроецировать вектор \(z\) сначала на подпространство, а потом
проекцию \(\hat z\) раскладывать по базису в подпространстве. Наш базис
ортогональный, поэтому фактически на втором шаге мы проецируем вектор
\(\hat z\) на базисные векторы. По теореме о трёх перпендикулярах можно
сразу проецировать \(z\) на базисные векторы :)

Как бы нам найти проекцию \(\hat z\) произвольного вектора \(z\) на
известный вектор \(v\)?

Вспомним, что скалярное произведение \(v'z\) --- это произведение длин
\(||v||\) и \(||z||\) на косинус угла между ними. Другими словам,
\(v'z\) --- это произведение длины \(||v||\) на длину проекции вектора
\(z\) на вектор \(v\): \[
v'z = ||v||\cdot ||z||\cdot \cos(v, z) = ||v||\cdot ||\hat z||
\]

Длина проекции \(\hat z\) вектора \(z\) равна: \[
||\hat z|| = \frac{v'z}{||v||}
\]

Осталось домножить длину проекции на вектор единичной длины в нужном
направлении: \[
\hat z = \frac{v}{||v||} \cdot ||\hat z|| = \frac{v}{||v||} \cdot \frac{v'z}{||v||} =  \frac{v'z}{v'v} \cdot v
\]

Например, проекция \(z\) на \((1, 1, -2, 0, 0)\) равна: \[
\frac{v'z}{v'v} \cdot \begin{pmatrix}
1 \\
1 \\
-2 \\
0 \\
0 \\
\end{pmatrix} = 
\frac{z_1 + z_2 - 2z_3}{2+2^2} \cdot \begin{pmatrix}
1 \\
1 \\
-2 \\
0 \\
0 \\
\end{pmatrix}
\]

А квадрат длины проекции \(\hat z\) вектора \(z\) на вектор \(v\) равен:

\[
||\hat z||^2 = \left(\frac{v'z}{||v||}\right)^2 = \frac{(v'z)^2}{v'v}
\]

Например, квадрат длины проекции \(z\) на \((1, 1, -2, 0, 0)\) равен: \[
(z_1 + z_2 - 2z_3)^2/(2+2^2);
\]

Мы разложили проекцию \(\hat z\) на сумму проекций!

\begin{multline}
\nonumber
\begin{pmatrix}
z_1 - \bar z \\
z_2 - \bar z \\
z_3 - \bar z \\
z_4 - \bar z \\
z_5 - \bar z \\
\end{pmatrix} = \\
= \frac{z_1 - z_2}{1+1^2}\begin{pmatrix}
1 \\
-1 \\
0 \\
0 \\
0 \\
\end{pmatrix} + 
\frac{z_1 + z_2 - 2z_3}{2+2^2}\begin{pmatrix}
1 \\
1 \\
-2 \\
0 \\
0 \\
\end{pmatrix} + 
\frac{z_1 + z_2 + z_3 - 3z_4}{3+3^2}\begin{pmatrix}
1 \\
1 \\
1 \\
-3 \\
0 \\
\end{pmatrix} + 
\frac{z_1 + z_2 + z_3 + z_4 - 4z_5}{4+4^2}\begin{pmatrix}
1 \\
1 \\
1 \\
1 \\
-4 \\
\end{pmatrix}
\end{multline}

Раз уж мы проецировали на ортогональные векторы, то по теореме Пифагора,
квадрат длинны всей проекции \(\hat z\) раскладывается на сумму
квадратов отдельных составляющих:

\begin{multline}
\nonumber
\sum_{i=1}^5 (z_i - \bar z)^2 = \\
= \frac{(z_1 - z_2)^2}{1+1^2} + \frac{(z_1 + z_2 - 2z_3)^2}{2+2^2} + \frac{(z_1 + z_2 + z_3 - 3z_4)^2}{3+3^2} + \frac{(z_1 + z_2 + z_3 + z_4 - 4z_5)^2}{4+4^2}
\end{multline}

В общем случае мы получим формулу:

\begin{multline}
\nonumber
\sum (z_i - \bar z)^2 = \\
= \frac{(z_1 - z_2)^2}{1+1^2} + \frac{(z_1 + z_2 - 2z_3)^2}{2+2^2} + \frac{(z_1 + z_2 + z_3 - 3z_4)^2}{3+3^2} + \ldots + \frac{(z_1 + z_2 + z_3 + \ldots + z_{n-1} - (n-1)z_n)^2}{(n-1)+(n-1)^2}
\end{multline}

В этом разложении явно видна сумма \((n-1)\) слагаемого. Каждое
слагаемое является квадратом нормальной стандартной случайной величины и
слагаемые независимы.

Можно и без геометрических соображений просто раскрыть скобки и по
индукции доказать равенство правой и левой части. Но там скучно :)

\subsection{Выборочная дисперсия --- матрицы}\label{--}

\paragraph{Вектор средних}\label{-}

Вектор средних в матричном виде записывается так:

\[
\hat z =
\begin{pmatrix}
\bar z \\
\bar z \\
\bar z \\
\bar z \\
\bar z \\
\end{pmatrix} = 
Hz=
\begin{pmatrix}
1/5 & 1/5 & 1/5 & 1/5 & 1/5 \\
1/5 & 1/5 & 1/5 & 1/5 & 1/5 \\
1/5 & 1/5 & 1/5 & 1/5 & 1/5 \\
1/5 & 1/5 & 1/5 & 1/5 & 1/5 \\
1/5 & 1/5 & 1/5 & 1/5 & 1/5 \\
\end{pmatrix} \cdot 
\begin{pmatrix}
z_1 \\
z_2 \\
z_3 \\
z_4 \\
z_5 \\
\end{pmatrix}
\]

Здесь \(\hat z\) --- это проекция \(z\) на подпространство
\(\Lin((1, 1, 1, 1, 1))\). В данном случае ранг матрицы \(H\) равен 1, в
ней одна строка повторяется кучу раз. Поэтому квадрат длины проекции,
\(||\hat z||^2\), имеет хи-квадрат распределение с одной степенью
свободы:

\[
||\hat z||^2 = n(\bar z)^2 \sim \chi^2_1
\]

\paragraph{Вектор отклонений от среднего}\label{---}

Вектор средних в матричном виде записывается так:

\[
z - \hat z =
\begin{pmatrix}
z_1 - \bar z \\
z_2 - \bar z \\
z_3 - \bar z \\
z_4 - \bar z \\
z_5 - \bar z \\
\end{pmatrix} = 
(I-H)z=
\left(
\begin{pmatrix}
1 & 0 & 0 & 0 & 0 \\
0 & 1 & 0 & 0 & 0 \\
0 & 0 & 1 & 0 & 0 \\
0 & 0 & 0 & 1 & 0 \\
0 & 0 & 0 & 0 & 1 \\
\end{pmatrix} -
\begin{pmatrix}
1/5 & 1/5 & 1/5 & 1/5 & 1/5 \\
1/5 & 1/5 & 1/5 & 1/5 & 1/5 \\
1/5 & 1/5 & 1/5 & 1/5 & 1/5 \\
1/5 & 1/5 & 1/5 & 1/5 & 1/5 \\
1/5 & 1/5 & 1/5 & 1/5 & 1/5 \\
\end{pmatrix} 
\right)
\cdot 
\begin{pmatrix}
z_1 \\
z_2 \\
z_3 \\
z_4 \\
z_5 \\
\end{pmatrix}
\]

Здесь \(z - \hat z\) --- это проекция \(z\) на подпространство
\(\Linp((1, 1, 1, 1, 1))\). В данном случае ранг матрицы \(I-H\) равен
\((n-1)\). Вместо ранга легче посчитать след --- сумму диагольных
элементов. Поэтому квадрат длины проекции, \(||\hat z||^2\), имеет
хи-квадрат распределение с одной степенью свободы:

\[
||z - \hat z||^2 = \sum_{i=1}^n (z_i - \bar z)^2 \sim \chi^2_{n-1}
\]

\subsection{Сумма квадратов остатков - геометрия}\label{-----}

Для максимальной доступности доказательства мы проведём его для двух
регрессоров. Случай \(k\) регрессоров ничем с геометрической точки
зрения не отличается.

Пусть \(y = \beta_x x + \beta_z z + u\), где \(u\) --- нормальный
вектор, \(u \sim \cN (0; \sigma^2 I)\).

Метод наименьших квадратов проецирует \(y\) на подпространство
регрессоров \(\Lin(x, z)\). Поэтому:

\begin{itemize}
\tightlist
\item
  вектор прогнозов \(\hat y\) --- проекция \(y\) на \(\Lin(x, z)\);
\item
  вектор ошибок \(\hat u = y - \hat y\) --- проекция \(y\) на
  \(\Lin(x, z)\);
\end{itemize}

Сумма квадратов остатков \(RSS = \sum \hat u^2\). По смыслу \(RSS\) ---
это квадрат длины проекции вектора \(y\) на \(\Linp(x, z)\). Замечаем,
что проекция \(x\) и \(z\) на \(\Linp(x, z)\) равна нулю, поэтому
\(RSS\) --- это проекция \(u\) на \(\Linp(x, z)\).

Чтобы сделать из вектора \(u\) стандартный нормальный вектор, достаточно
поделить его на \(\sigma\). При этом квадрат длины проекции поделится на
\(\sigma^2\). Мы увидели, что \(RSS/\sigma^2\) --- это квадрат длины
проекции стандартного нормального вектора на подпространство
\(\Linp(x, z)\) размерности \((n-2)\). Итого, \emph{по определению}
\(RSS/\sigma^2 \sim \chi^2_{n-2}\).

\subsection{Сумма квадратов остатков - матрицы}\label{-----}

Для полноты картины осталось лишь сказать, что вектор прогнозов
представим в виде:

\[
\hat y = H y = \beta_x Hx + \beta_z Hz + H u = \beta_x x + \beta_z z + H u,
\] где \(H = X(X'X)^{-1}X'\), а матрица \(X\) имеет вид

\[
\begin{pmatrix}
x_1 & z_1 \\
x_2 & z_2 \\
x_3 & z_3 \\
\vdots & \vdots \\
x_n & z_n \\
\end{pmatrix}
\]

А вектор остатков, соответственно, представим в виде:

\[
\hat u = (I-H)y = (I-H)u,
\] где \(H = X(X'X)^{-1}X'\).

И сумма квадратов остатков, по-прежнему, есть квадрат длины вектора
\(\hat u\), и может быть записана в матрицах как: \[
RSS = \hat u' \hat u = ((I-H)u)'(I-H)u = u'(I-H)'(I-H)u = u'(I-H)u,
\]

Ранг матрицы \(M=I-H\) равен её следу, так как матрица \(M=I-H\) тоже
проецирует векторы. Осталось найти размерность подпространства:

\[
\rang H = \trace (I-H)= n - \trace H = n - \trace (X(X'X)^{-1}X') = n - \trace((X'X)^{-1}X'X)= n - \trace(I_{2\times 2})= n -2
\]

\subsection{Про t и F отдельно :)}\label{-t--f-}

Раз уж хи-квадрат --- это квадрат длины проекции, то с точностью с
домножения на размерность подпространства окажется, что
\(t\)-распределение --- это тангенс, а \(F\)-распределение --- квадрат
тангенса угла.

Но об этом подробнее в продолжении :)

\subsection{Пояснительная записка}\label{-}

Сразу скажем, что этот подход не нов. Например, он обсуждается в статье
\textcite{cobb2011teaching}. Однако аккуратного изложения его на русском
я не знаю, поэтому и появилась эта заметка :)

Достоинство подхода с определением через квадрат длины проекции состоит
в том, что доказательство многих теорем значительно облегчается. Да и
геометрическая мотивация лучше просто формулы!

Возникает разумный вопрос, почему-бы не дать стандартное определение и
дополнить его теоремой о том, что квадрат длины проекции имеет
хи-квадрат распределение? На мой взгляд, это лучше, чем текущее
положение дел, но всё равно неидеально. При подходе ``стандартное
определение + теорема'' получается, что даётся определение, которое не
используется. Поэтому определять хи-квадрат распределение нужно как
квадрат длины проекции нормального стаднартного вектора. А дальше
доказывать непротиворечивость этого определения для желающих.

\printbibliography


\end{document}
