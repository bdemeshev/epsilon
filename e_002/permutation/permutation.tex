\documentclass[10pt]{article}

\usepackage{epsilonj}

\usepackage{tikz}
\usetikzlibrary{calc,intersections,decorations.pathreplacing} 
\usetikzlibrary{arrows.meta}
% этот код считает отмечает углы в tikz
\newcommand\markangle[6]{% origin X Y radius radiusmark mark
  % fill red circle
  \begin{scope}
    \path[clip] (#1) -- (#2) -- (#3);
    \fill[color=red,fill opacity=0.5,draw=red,name path=circle]
    (#1) circle (#4);
  \end{scope}
  % middle calculation
  \path[name path=line one] (#1) -- (#2);
  \path[name path=line two] (#1) -- (#3);
  \path[%
  name intersections={of=line one and circle, by={inter one}},
  name intersections={of=line two and circle, by={inter two}}
  ] (inter one) -- (inter two) coordinate[pos=.5] (middle);
  % put mark
  \node at ($(#1)!#5!(middle)$) {#6};
}

\RequirePackage{graphicx}
\RequirePackage[colorlinks,citecolor=blue,urlcolor=blue]{hyperref}

\newcommand{\specialcell}[2][c]{\begin{tabular}[#1]{@{}c@{}}#2\end{tabular}}
\newcommand{\RR}{\mathbb{R}}

\begin{document}

\TITLE{Случайная перестановка (рабочее название)}
\SHORTTITLE{Случайная перестановка}

\AUTHOR{Борис Демешев}{НИУ ВШЭ, Москва.}
\SHORTAUTHOR{Борис Демешев}

\DoFirstPageTechnicalStuff


\newtheorem{theorem}{Теорема}
\newtheorem{definition}{Определение}

\begin{abstract}
Случайная перестановка
\end{abstract}

\begin{keyword}
задача, случайная перестановка, киллер
\end{keyword}



\section{Сумасшедшая старушка}

В самолете $100$ мест и все билеты проданы. Первой в очереди на посадку стоит Сумасшедшая Старушка. Сумасшедшая Старушка очень переживает, что ей не хватит места, врывается в самолёт и несмотря на номер по билету садиться на случайно выбираемое место. Каждый оставшийся пассажир садится на своё место, если оно свободно, и на случайное выбираемое место, если его место уже кем-то занято.

\begin{enumerate}
% \item Какова вероятность того, что все пассажиры сядут на свои места?
%\item Какова вероятность того, что второй пассажир в очереди сядет на своё место? 
\item Какова вероятность того, что последний пассажир сядет на своё место?
\item Чему примерно равно среднее количество пассажиров севших на свои места?
\end{enumerate}




\section{Судьба Дон-Жуана} 

У Дон-Жуана $n$  знакомых девушек, и их всех зовут по-разному. Он пишет
им $n$  писем, но по рассеянности раскладывает их в конверты
наугад. Случайная величина $X$ обозначает количество девушек, получивших письма, адресованные лично им.

\begin{enumerate}
\item Найдите $\E(X)$, $\Var(X)$
\item Какова при большом $n$ вероятность того, что хотя бы одна девушка получит письмо, адресованное ей?
\end{enumerate}


\section{Киллер}

Правила игры «Киллер» просты. Игроки пишут на бумажках, как их зовут, и кладут бумажки в шляпу. Каждый тянет из шляпы имя своей первой жертвы. Если первой жертвой игрока является он сам, то он совершает «самоубийство» и дальше не играет\footnote{В некоторых вариантах правил, если игрок вытянул из шляпы своё имя, то он должен вытянуть другую бумажку.}. Чтобы убить жертву, надо остаться с ней наедине и сказать: «Ты убит!». Убийца забирает себе все бумажки, набранные убитым, и начинается охотиться за тем, за кем охотился убитый.  Побеждает тот, кто наберёт больше всех бумажек к концу игры. Заметим, что в «Киллере» каждый игрок оказывается втянут в одну из нескольких цепочек.

В «Киллера» играют 30 человек, из них 20 девушек. 

\begin{enumerate}
\item Какова вероятность того, что в цепочке, начинающейся с Маши Сидоровой ровно 5 человек?
\item Какова вероятность того, что в цепочке, начинающейся с Маши Сидоровой ровно 5 девушек?
\item Какова вероятность того, что все девушки попадают в одну цепочку убийц и жертв?
\item Какова вероятность того, что все игроки попадают в общую цепочку?
\item Сколько в среднем цепочек в «Киллере»?
\item Сколько в среднем «самоубийц»?
\end{enumerate}


\section{Ключи и копилки}

На столе стоят $n$ свиней-копилок. Достать содержимое копилки можно двумя
способами: либо разбить копилку, либо открыть дно специальным
ключиком. К каждой копилке подходит единственный ключ. Мы раскладываем ключи по
копилкам наугад, один ключ в одну копилку. Затем разбиваем $k$ копилок и получаем хранящиеся в них ключи. Далее мы будем копилки только открывать ключами.
\begin{enumerate}
\item Какова вероятность того, что мы сможем достать все ключи? 
\item Какая доля ключей в среднем будет найдена?
\end{enumerate}

\section{Задача о макаронах}

В тарелке запутавшись лежат $n>>0$ макаронин. Я по очереди связываю попарно все торчащие концы макаронин. 

\begin{enumerate}
\item Какова примерно вероятность того, что я свяжу все макаронины в одно большое кольцо?
\item Сколько в среднем колец образуется?
\item Каково среднее число колец длиной в одну макаронину?
\end{enumerate}

\section{Задача о 100 заключенных}

У ста узников тюрьмы есть последний шанс на спасение. В комнате стоит шкаф в котором сто занумерованных ящичков. Палач кладёт в каждый ящичек бумажку с номером ровно одного из заключенных в случайном порядке и задвигает все ящички. Узники заходят в комнату один за другим. Каждый узник может открыть любые 50 ящичков. После каждого узника все ящички задвигаются в исходное положение. Если каждый узник находит свой номер, то все узники будут помилованы. Если хотя бы один из узников не найдёт свой номер, то все будут казнены. 

Узники могут предварительно договориться о стратегии.

\begin{enumerate}
\item Какова оптимальная стратегия?
\item Какую вероятность выигрыша она обеспечивает?
\end{enumerate}
  

\section{что-то про детерминант?}








\end{document}