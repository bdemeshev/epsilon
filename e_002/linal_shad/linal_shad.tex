\documentclass[10pt]{article}

\usepackage{epsilonj}

\RequirePackage{graphicx}
\RequirePackage[colorlinks,citecolor=blue,urlcolor=blue]{hyperref}

\newcommand{\specialcell}[2][c]{\begin{tabular}[#1]{@{}c@{}}#2\end{tabular}}

\begin{document}

\TITLE{ШАД и линал}
\SHORTTITLE{ШАД и линал}

\AUTHOR{Артём Филатов}{НИУ ВШЭ, Москва.}
\SHORTAUTHOR{Филатов А.}

\DoFirstPageTechnicalStuff

\begin{abstract}
\end{abstract}

\begin{keyword}
ШАД, линейная алгебра
\end{keyword}

\section{Кратко про шад}

В 2007 году компания Яндекс основала в своих стенах Школу Анализа Данных. Школа была создана для подготовки специалистов в области анализа больших данных, машинного обучения и других смежных дисциплин. Ежегодно в апреле начинаются экзамены, которые проходят в три этапа: онлайн -- тест, письменный экзамен и собеседование. Письменный экзамен включает в себя задачи по теории вероятности, алгоритмам, линейной алгебре, математическому анализу и комбинаторике. Предлагаю вам разбор нескольких занимательных задач по линейной алгебре из письменных экзаменов прошлых лет.

\section{Задачи по линейной алгебре из шАДовских экзаменов}

Задача №1

Дана матрица $A$ размера $n \times n$, где $a_{i,j} = (i-j)^2, i,j = 1,\ldots ,n$. Найдите ранг матрицы $A$.

\textit{Решение:}

Посмортим, как выглядит наша матрица.
\[
A = \begin{pmatrix}
0 & 1 & 4 & \cdots & (n-2)^2 & (n-1)^2 \\
1 & 0 & 1 & 4 & \cdots & (n-2)^2 \\
4 & 1 & 0 & 1 & \cdots & (n-3)^2 \\
9 & 4 & 1 & 0 & \cdots & (n-4)^2 \\
\cdots & \cdots & \cdots & \cdots & \cdots & \cdots \\
(n-1)^2 & (n-2)^2 & (n-3)^2 & (n-4)^2 & \cdots & 0
\end{pmatrix}
\]

Из условия каждый элемент матрицы $A$ равен $(i-j)^2 = i^2 - 2ij + j^2$. Но у матрицы из элементов $i^2$ ранг 1, у матрицы из элементов $j^2$ ранг тоже единица. Посмотрим на матрицу, образованную $ij$:

\[
\begin{pmatrix}
1 & 2 & 3 & \cdots & n \\
2 & 4 & 6 & \cdots & 2n \\
3 & 6 & 9 & \cdots & 3n \\
\cdots & \cdots & \cdots & \cdots \\
\end{pmatrix}
\]
Ее ранг также не превосходит 1. Нам известно, что $\rank(A + B) \leq \rank(A) + \rank(B)$, следовательно $\rank(A) \leq 3$. Но можно показать, что у нас есть ненулевые миноры 3 порядка, следовательно $\rank(A) = 3$.

\hspace{2mm}

Задача №2

Дана матрица из нулей и единиц, причем для каждой строки матрицы верно следующее: если в строке есть единицы, то они все идут подряд. Докажите, что определитель такой матрицы равен 0 или $\pm 1$.

\textit{Решение:}

Посмротрим на то, как выглядит одна из наших матриц:

\[
A = \begin{pmatrix}
0 & 1 & 1 & 1 & 0 & 0 \\
1 & 1 & 1 & 0 & 0 & 0 \\
0 & 1 & 0 & 0 & 0 & 0 \\
0 & 0 & 0 & 1 & 1 & 1 \\
0 & 0 & 1 & 1 & 0 & 0 \\
1 & 1 & 1 & 0 & 0 & 0
\end{pmatrix}
\]

Переставим строки так, чтобы образовать некое подобие ступенчатой матрицы.

\[
A = \begin{pmatrix}
1 & 1 & 1 & 0 & 0 & 0 \\
1 & 1 & 1 & 1 & 0 & 0 \\
0 & 1 & 1 & 1 & 0 & 0 \\
0 & 1 & 0 & 0 & 0 & 0 \\
0 & 0 & 1 & 1 & 0 & 0 \\
0 & 0 & 0 & 1 & 1 & 1
\end{pmatrix}
\]

Что произойдет с определителем? Он либо не изменился, либо изменил знак, так как перестановка строк меняет знак определителя на противоположный. Теперь сделаем следующее: если позиции первых единиц у строк совпали, то вычтем из той в которой больше единиц, ту в которой меньше единиц. На определитель данное преобразование никак не влияет.

\[
A = \begin{pmatrix}
1 & 1 & 1 & 0 & 0 & 0 \\
0 & 0 & 0 & 1 & 0 & 0 \\
0 & 0 & 1 & 1 & 0 & 0 \\
0 & 1 & 0 & 0 & 0 & 0 \\
0 & 0 & 1 & 1 & 0 & 0 \\
0 & 0 & 0 & 1 & 1 & 1
\end{pmatrix}
\]

Переставляя строки и повторяя данную процедуру, мы получим ступенчатую матрицу, которая будет либо вырождена, либо иметь единицы на диагонали. А так как в такой матрице $\det(A) = \prod_{i=1} a_{i,i} = 1$, то детерминант исходной матрицы равен либо 0, либо $\pm 1$.

\hspace{2mm}

Задача №3

Опишите все невырожденные вещественные матрицы $A$, для которых все элементы матриц $A$ и $A^{-1}$ неотрицательны.

\textit{Решение:}

Пусть исходная невырожденная матрица A заполнена некотрыми элементами $a_{i,j}$, а обратная к ней $A^{-1}$ элементами $b_{i, j}$. Как известно, $AA^{-1} = E$. Значит, произведение первой строки на первый столбец должно дать 1:

\[
a_{1,1}\cdot b_{1,1} + a_{1,2}\cdot b_{2,1} + \cdots + a_{1,n}\cdot b_{n,1} = 1 
\]

Но произведение первой строки на все остальные столбцы должно дать 0, также нам известно, что все элементы матриц неотрицательны, значит если $a_{1,i} \neq 0$, то $b_{i,j} = 0, j = 2\ldots n$. Это должно быть выполнено для всех $a_{i,j}$. Формально: 

\[a_{i,j} \neq 0 \Rightarrow b_{j,z} = 0, z = 1\ldots (i-1),(i+1)\ldots n\]

 
 Докажем, что нет такой матрицы $A$ с двумя и более положительными элементами в одном столбце:
 
 \vspace{2mm}
 
 Зафиксируем столбец $j$. Предположим мы встретили первый ненулевой элемент, тогда все кроме одного элементы в $j$ строке матрицы $A^{-1}$ равны 0. Предположим, что мы встретили второй положительны элемент, тогда он занулит все элементы кроме одного, включая тот, который сы не занулили в первый раз. Следовательно, мы получили, что $b_{j,z} = 0$ для всех $z = 1\ldots n$.
 Но это невозможно, так как это означало бы, что все алгебраические дополнения в некой строке матрицы $A$ равны 0 ($b_{i,j} = \frac{Alg_{i,j}}{\det(A)}$), а следовательно и определитель. 

Из всего сказанного следует, что единственно законной матрицей $A$ будет такая матрица, в столбцах которых по одному положительному элементу. Элементарными преобразованиями такая матрица приводится к диагональному виду. Мы показали, что все элементы обратной матрицы зануляться, кроме тех, которые образуют 1 в произведении с ненулевыми элементами матрицы $A$, следовательно обратная матрица будет иметь аналогичный вид. 

\hspace{2mm}

Задача №4

Имеется некоторый ненулевой вектор -- столбец $v$. Найти все собственные значения матрицы $v\cdot v^T$.

\textit{Решение:}

Первым делом необходимо понять сколько собственных значений нужно найти. Нам известно, что $\rank(A) \cdot \rank(B) \leq \min {\rank(A), \rank(B)}$. Следовательно итоговая матрица будет иметь ранг равный единице. Ранг при замене базиса не изменяется, тогда мы можем перейти к диагональному виду матрицы с базисом из собственных векторов, где на диагонали будет лишь одно собственное значение. Собственные значения также не изменяются при замене базиса. Осталось его найти! 

Для того, чтобы найти собственное значение (можно дагодаться чему оно равно) воспользуемся еще одним интресеным свойством. Оказывается, что след матрицы (сумма диагональных элементов) равен сумме собствееных значений матрицы с учетом кратности. Легко увидеть, что сумма диагональных элементов это скалярное произведение вектора на самого себя.

Следовательно, мы имеем собственное значение $v^T\cdot v$, и нулевые собственные значения кратности $n - 1$.

\end{document}