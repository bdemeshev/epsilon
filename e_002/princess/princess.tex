\documentclass[10pt]{article}

\RequirePackage{verse} % load it before hyperref!!!

\usepackage{epsilonj}



\RequirePackage{graphicx}
\RequirePackage[colorlinks,citecolor=blue,urlcolor=blue]{hyperref}

\newcommand{\specialcell}[2][c]{\begin{tabular}[#1]{@{}c@{}}#2\end{tabular}}
\newcommand{\RR}{\mathbb{R}}

\begin{document}

\TITLE{Доказательство от принцессы}
\SHORTTITLE{Доказательство от принцессы}

\AUTHOR{Борис Демешев}{НИУ ВШЭ, Москва.}
\SHORTAUTHOR{Борис Демешев}

\DoFirstPageTechnicalStuff


\newtheorem{theorem}{Теорема}
\newtheorem{definition}{Определение}

\begin{abstract}
Доказательство от принцессы --- частный случай доказательства от противного
\end{abstract}

\begin{keyword}
доказательство от противного, принцесса, доказательство
\end{keyword}

% from http://tex.stackexchange.com/questions/99092/
\newcommand{\attrib}[1]{\nopagebreak\noindent
  \makebox[\dimexpr(\versewidth)][r]{\footnotesize#1}}

% задаём ширину по самой широкой строке в стихе:
\settowidth\versewidth{Как печальна принцесса\ldots Что бы значило это?}

\begin{flushright}
\parbox{\versewidth}{
Как печальна принцесса\ldots Что бы значило это? \\
Ее губы поблекли; сердце скорбью одето:  \\
улыбается грустно; вздох уныл и глубок\ldots \\
В золотом ее кресле с ней тоска неразлучна, \\
и замолк клавесина аккорд полнозвучный, \\
и цветок позабытый увядает у ног.  \\
\attrib{Сонатина, Рубен Дарио}
}
\end{flushright}




\section{Доказательство от противного}

Как устроено классическое доказательство от противного? Берём кого-нибудь противного, и пусть он доказывает. Допустим нам нужно доказать, что утверждение А верно. Мы, наоборот, предполагаем, что А неверно. Далее каким-нибудь образом приходим к противоречию и, таким образом, получаем вывод, что наше допущение А было ложно.

Довольно часто доказательство от противного используется для того, чтобы доказать, что какой-нибудь объект $X$ не существует. В этом случае очень удобно использовать предлагаемое доказательство от принцессы. Мы представляем себе принцессу, которая замуж не хочет, а по традиции должна объявить конкурс для претендентов руку и сердце. И она объявляет: «Тот, кто принесёт мне $X$, сможет на мне жениться!». А дальше остаётся объяснить, как она будет аргументированно отказывать каждому претенденту.

\section{Пара примеров}

Классический пример доказательства от принцессы --- доказательство того, что максимальное простое число не существует. Принцесса объявляет: «Тот, кто принесёт мне самое большое просто число во Вселенной, получит меня в жёны!». И к примеру приходит принц и приносит ей $p_n$. А она ему в ответ: «Не пойду я за тебя замуж, ведь простое число
$p_1 \cdot p_2 \cdot p_3 \cdot \ldots \cdot p_n + 1$
больше чем ты принёс!» Так принцесса отказывает всем ухажёрам, а, следовательно, наибольшего простого числа не существует.

Идея доказательства от принцессы возникла так. Я иногда веду вводный курс стохастического анализа для экономистов. Если требуется и позволяет время, то рассказываю про мощности множеств и, в частности, про то, что множество последовательностей из 0 и 1 несчётное. И в нём есть один тонкий момент. Если проводить доказательство в общем виде с произвольными буквами, то оно слишком тяжеловесно. Если проводить на конкретном примере, то возникает вопрос, а почему это доказательство. И принцесса замечательно решает  проблему доказательства на частном примере!

Принцесса объявляет: «Тот, кто занумерует натуральными числами все бесконечные последовательности из 0 и 1, получит меня в жёны». И, к примеру, приходит принц датский и говорит: «Я занумеровал!» И предъявляет листочек, на котором все последовательности занумерованы:

\begin{enumerate}
\item 000000000\ldots
\item 011001010\ldots
\item 101000000\ldots
\item 010011010\ldots

\ldots
\end{enumerate}

Как принцессе отказать принцу датскому? Она выбирает диагональные элементы этих последовательностей 0110\ldots\, Затем меняет 0 на 1, а 1 на 0, получая 1001\ldots\, И спрашивает принца датского: «А последовательность 1001\ldots\, у Вас под каким номером?» И принц датский начинает перебирать. Под первым номером не может идти, так как первой цифрой отличается, под вторым номером не может идти, так как вторым номером отличается\ldots\, И принц датский трагично вынужден признать, что эту последовательность он забыл занумеровать. И подобным образом принцесса сможет отказать всем претендентам, а значит множество последовательностей несчётно.



\end{document}
