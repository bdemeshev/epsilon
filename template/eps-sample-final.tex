\documentclass[final,pdftex]{epsilonj}

\RequirePackage{graphicx}
%\usepackage{microtype}

\RequirePackage[colorlinks,citecolor=blue,urlcolor=blue]{hyperref}

\startlocaldefs
\numberwithin{equation}{section}
\endlocaldefs

\addbibresource{epsilon1.bib}


\begin{document}

% \microtypesetup{protrusion=false, expansion=false}
\begin{frontmatter}
\title{Образец статьи: новый подход к вёрстке русскоязычных журналов\protect\thanksref{T1}}
\runtitle{Новый подход к вёрстке русскоязычных журналов}
\thankstext{T1}{Ссылка к заголовку от команды thankstext.}

\begin{aug}
\author{\imya{Первоксандр} \fam{Авторов}\thanksref{t1,t2}\ead[label=e1]{elektronnaya@pochta.ru}}%
\author{\imya{Втордимир} \fam{Авторский}\thanksref{t3}\ead[label=e2]{adres@pisem.net}}%
\author{\imya{Третьексей} \fam{Авторчук}
\ead[label=e3]{third@somewhere.com}
\ead[label=u1,url]{www.gde-to.ru}}

\thankstext{t1}{Какой-то комментарий}
\thankstext{t2}{Первый сотрудник проекта}
\thankstext{t3}{Второй сотрудник проекта}
\runauthor{П. В. Авторов и др.}

\address{Адрес первого и второго автора,
как правило, в пару строк\\
\printead{e1,e2}}

\address{Адрес третьего автора,
как правило, в пару строк,
обычно в одну-две\\
\printead{e3,u1}}
\end{aug}

\begin{abstract}
Аннотация должна передавать краткое содержание работы.
Она должна быть ясной, содержательной, релевантной и~короткой
(не более 150~слов). Аннотация должна содержать информацию,
необходимую для поиска по базам научных работ. 
В~аннотации не должно быть математических формул.

Этот файл является образцом. Сравните его исходный код
с~финальным PDF-файлом, чтобы получить представление о~том,
как написать статью по данному шаблону.
\end{abstract}

\begin{keyword}
\kwd{образец}
\kwd{написание статей}
\kwd{\LaTeX}
\kwd{эпсилон}
\end{keyword}

\end{frontmatter}

% \microtypesetup{protrusion=true, expansion=true}

\section{Обычный текст}

В конце   предложений принято ставить пробел. Неважно, сколько
пробелов будет поставлено при наборе, что один, что сотня.
Новая строка (одинарная) считается пробельным символом.

Одна   или более пустых строк обозначают конец абзаца.

Так как любое количество последовательных пробелов считывается как
один, форматирование исходного текстового файлы безразлично для
системы 
      \TeX,         % Команда \TeX рисует логотип TeX’а
поэтому пользователь волен украшать свой файл пробелами так,
как ему угодно.
Когда вы используете
      \LaTeX,       % Команда \LaTeX рисует логотип LaTeX’a
создание входного файла, который легко читать, сильно облегчает
работу редактора (и, как ни странно, самого автора).
В исходном тексте данного файла показано, как можно вставлять комментарии в текст; они не будут отображаться при выводе, точно так же как комментарии в коде программ не играют роли при компиляции, однако облегчают чтение.

Исходный текст предыдущего абзаца содержит множество одинарных разрывов строк, однако более целесообразно вводить текст по принципу «один абзац "--- одна строка».

Так как вывод на экран отличается от того, что вводится с клавиатуры, некоторые вещи делаются при помощи специальных команд. Некоторые символы отсутствуют на стандартной клавиатуре, однако существуют типографские раскладки, позволяющие их ввести напрямую (очень рекомендуется, например, использовать при наборе раскладку, доступную по \href{http://ilyabirman.ru/projects/typography-layout/}{данной ссылке}). 
«Такие» кавычки можно ввести при помощи комбинации клавиш \texttt{Right Alt+<} и \texttt{Right Alt+>}, однако если возможности поставить раскладку нет, то <<ёлочки>> можно ввести посредством двух знаков <<меньше>>. Точно так же вводятся «внутренние „кавычки-лапки“»: либо напрямую с клавиатуры через \texttt{Right Alt+Shift+<} и~\texttt{Right Alt+Shift+>}, либо через ,,две запятые и~два обратных апострофа``.

Существует четыре вида «чёрточек»: дефис, среднее тире, длинное тире и минус. Дефис ставится в сложных словах (как-то так), среднее тире используется в интервалах чисел (3--6~месяцев), а тире "--- это знак, разделяющий части предложения. Минус же используется в математических выражениях (корень равен $-3$). Базовая система \LaTeX{} поддерживает международное длинное тире (em dash --- такое), однако в русскоязычных текстах используется три вида длинных тире: такое "--- с обычных предложениях, вот такое в названиях совместных открытий (преобразование Бокса"--~Кокса) и такое (см. ниже) в диалогах.

"--* Это больше применимо к художественной литературе, чем к научным текстам, однако это следует знать всем, кто пишет в \LaTeX'е.

В английских текстах пробел после точки в конце предложения длиннее, чем остальные пробелы. В русском языке такого правила нет. После всех точек ставится обычный пробел, кроме тех, которые являются частью общеупотребительных сокращений, т.\,е. в них ставится тонкий пробел, т.\,н. «тонкая шпация». Чтобы поставить многоточие, используйте специальную команду (предпочтительно специально написанную для данного стилевого файла\ldotst{} однако сойдёт и такая\ldots{}). С другой стороны, в научных текстах многоточие "--- редкий гость, а те, что присутствуют в формулах, следует набирать общепринятой командой ($x_1, \ldots, x_n$).

\TeX{} интерпретирует некоторые специальные символы  как команды, поэтому их надо вводить особым образом. К ним относятся 
       \$ \& \% \# \{ \}.

Чтобы выделить фрагмент курсивом, используется следующая \textit{команда}.


\emph{Большой кусок текста можно выделить курсивом следующим образом. Отдельные слова внутри таких текстов выделяются посредством \emph{отмены курсива}. Однако следует помнить, что в тексте, в котором выделено всё, не выделено ничего, поэтому не следует прибегать к столь интенсивному выделению.}

В типографской практике не принято разрывать строку после некоторых символов. Однобуквенные слова на конце строк смотрятся очень плохо, поэтому \TeX'ом предусмотрен специальный неразрывный пробел (обычная тильда). Также следует связывать такие конструкции, как «переменная~$x$», «33~измерения» и~«и~т.\,п.». Если требуется запретить перенос целого слова, для этого предусмотрена специальная команда (слово \mbox{\emph{ВНИИСОК}} не должно переноситься).

\TeX{} "--- это отличная среда для набора математических формул навроде 
       $ x-3y = 7 $
или
       $ a_{1} > x^{2n} / y^{2n} > x' $.
Помните, что если буква~$x$ является математическим символом, её нужно делать формулой.

\section{Сноски}
Сноски\footnote{Это первый образец сноски.}
не представляют ровно никакой проблемы\footnote{Ещё одна}\fnnsp. Сноски ставятся \emph{перед} запятой, точкой с запятой, двоеточием, тире и точкой, однако \emph{после} вопросительного и восклицательного знаков, многоточия и закрывающей кавычки. Если сновка стоит перед точкой или запятой\footnote{Вот так.}, в данном пакете прописана команда\footnote{Это происходит второй раз за весь абзац!}\fnnsp, позволяющая её сдвинуть, таким образом сделав её эстетичнее.

\section{Выключные элементы текста}

Выключку можно сделать, увеличив отступы с краёв текста. Как правило, цитаты оформляются именно таким способом.
Существуют короткие цитаты 
\begin{quote}
   Это короткая цитата, состоящая из одного абзаца. Абзацные отступы отсутствуют. 
\end{quote}
и длинные.
\begin{quotation}
   Это более длинная цитата. Она состоит из двух абзацев. В начале каждого из них делается красная строка.

   Это второй абзац цитаты. Он такой же скучный, как и первый.
\end{quotation}

Списки "--- это ещё одна часто используемая специальная структура текста.
Приводится пример \emph{маркированного} списка из двух уровней.
\begin{itemize}
\item Это первый элемент списка. Перед элементами ставится маркер типа «буллит». В других стилях этот элемент может меняться.

Элемент списка может содержать абзацный разрыв. Ничего страшного.
\item  Это второй элемент списка. Он содержит вложенный маркированный список. 
    \begin{itemize}
       \item Это первый элемент маркированного списка внутри маркированного списка. 
          \item Это второй элемент маркированного списка.  \LaTeX{} позволяет делать более глубокое вложение списков, чем это эстетически приемлемо.
      \end{itemize}
      А это конец второго элемента внешнего списка. Он такой же скучный, как и большинство примеров. 
   \item  Третий элемент.
\end{itemize}


Ниже приводится \emph{нумерованный} список. Списки могут быть до четырёх уровней вложенности.
\begin{enumerate}
	\item Это первый элемент списка. Перед элементами ставится маркер типа «буллит». В других стилях этот элемент может меняться.
	
	Элемент списка может содержать абзацный разрыв. Ничего страшного.
	\item  Это второй элемент списка. Он содержит вложенный маркированный список. 
	\begin{enumerate}
		\item Это первый элемент маркированного списка внутри маркированного списка. 
		\item Это второй элемент маркированного списка.  \LaTeX{} позволяет делать более глубокое вложение списков, чем это эстетически приемлемо.
	\end{enumerate}
	А это конец второго элемента внешнего списка. Он такой же скучный, как и большинство примеров. 
	\item  Третий элемент.
\end{enumerate}


Поддерживается даже поэзия.
\begin{verse}
   Могу стихотворенье написать я \\         % Команда \\ разделяет линии
   В среде одной, достойной проклятья,   % внутри строфы.

               % Пустые строки разделяют строфы.

   Поскольку эти строки \\
   Мы \emph{сами} бьём на блоки, \\
   В которых вообще можно написать больше слов, чем умещается без ущерба для эстетического восприятья!
\end{verse}

Математические формулы также принято делать выключками. Выключная формула (занимающая отдельную строку) выравнивается по центру. Для формул, занимающих несколько строк, существует специальная команда.
   \[  x' + y^{2} = z_{i}^{2}\]
Не следует начинать абзац с выключной формулы. Не следует делать выключную формулу отдельным абзацем.

Пример теоремы:


\begin{thm}
Все гипотезы интересны, однако некоторые гипотезы интереснее, чем другие. 
\end{thm}

\begin{proof}
Это очевидно.
\end{proof}

\section{Таблицы и изображения}
Ссылка на таблицу с меткой: как видно из таблицы~\ref{tab:sphericcase} на стр.~\pageref{tab:sphericcase}, а также из табл.~\ref{tab:parset} на стр.~\pageref{tab:parset}.


\begin{table*}
\begin{tabular}{crrrrc}
\toprule
Точки \\
равнов. & \multicolumn{1}{c}{$x$} & \multicolumn{1}{c}{$y$} & \multicolumn{1}{c}{$z$} & \multicolumn{1}{c}{$C$} &
S \\
\midrule
$L_1$ & $-$2.485252241 & 0.000000000 & 0.017100631 & 8.230711648 & U \\
$L_2$ &    0.000000000 & 0.000000000 & 3.068883732 & 0.000000000 & S \\
$L_3$ &    0.009869059 & 0.000000000 & 4.756386544 & $-$0.000057922 & U \\
$L_4$ &    0.210589855 & 0.000000000 & $-$0.007021459 & 9.440510897 & U \\
$L_5$ &    0.455926604 & 0.000000000 & $-$0.212446624 & 7.586126667 & U \\
$L_6$ &    0.667031314 & 0.000000000 & 0.529879957 & 3.497660052 & U \\
$L_7$ &    2.164386674 & 0.000000000 & $-$0.169308438 & 6.866562449 & U \\
$L_8$ &    0.560414471 & 0.421735658 & $-$0.093667445 & 9.241525367 & U \\
$L_9$ &    0.560414471 & $-$0.421735658 & $-$0.093667445 & 9.241525367 & U \\
$L_{10}$ & 1.472523232 & 1.393484549 & $-$0.083801333 & 6.733436505 & U \\
$L_{11}$ & 1.472523232 & $-$1.393484549 & $-$0.083801333 & 6.733436505 & U
\\ \bottomrule
\end{tabular}
\caption{Сферический случай ($I_1=0$, $I_2=0$)}
\label{tab:sphericcase}
\end{table*}

Большинство различий объясняется значениями частного коэффициента воспроизводства~$\pi$ для больших значений темпа роста~$\mu$. В ранних публикациях \cite{lobanov10,burmistrova10} было обнаружено, что высокая концентрация глюкозы коррелирует с низким выходом пенициллина. Недавние открытия \cite{lobanov10,burmistrova10,romanko02,bonar06} продемонстрировали, что высокая концентрация глюкозы замедляет синтез энзимов.

Это не противоречит результатам работы \cite{risch70}, в которой рассматривается непрерывная ферментация белковой культуры. Параметр~$\mu$ описывается кинетикой
\begin{equation}\label{eq:cs}
    C_{s}  =  K_{M} \frac{\mu/\mu_{x}}{1-\mu/\mu_{x}} 
\end{equation}
Pirt \& Rhigelato вычислили $\pi$ для $\mu$ от~$0.023$ до~$0.086$~h$^{-1}$. Они также приводят значение $\mu_{x} \approx 0.095$ для~h$^{-1}$. Из уравнения~\eqref{eq:cs} видно, что $K_{M}=1$. Однако экспериментальных данных, опубликованных Ryu \& Hospodka, недостаточно для подтверждения гипотезы о том, что результаты анализа идентичны.

\begin{table}
\centering
\begin{tabular}{lrll}
\toprule
\multicolumn{2}{l}{Параметр} & {\it Набор 1} & {\it Набор 2}\\
\midrule
$\mu_{x}$           & [ч$^{-1}$]  & 0.092       & 0.11          \\
$K_{x}$             & [г/г]     & 0.15        & 0.006         \\
$\mu_{p}$           & [г/г]  & 0.005       & 0.004         \\
$K_{p}$             & [г/л]        & 0.0002      & 0.0001        \\
$K_{i}$             & [г/л]        & 0.1         & 0.1           \\
$Y_{x/s}$           & [г/г]     & 0.45        & 0.47          \\
$Y_{p/s}$           & [л/г]        & 0.9         & 1.2           \\
$k_{h}$             & [ч$^{-1}$]  & 0.04        & 0.01          \\
$m_{s}$             & [м/с${}^2$]  & 0.014       & 0.029         \\
\bottomrule
\end{tabular}
\caption{Набор параметров из работы Bajpai \& Reu\ss}\label{tab:parset}
\end{table}

Bajpai \& Reu\ss\ decided to disregard the
differences between time constants for the two regulation mechanisms
(glucose repression or inhibition) because of the
relatively very long fermentation times, and therefore proposed a Haldane
expression for $\pi$.


\begin{figure} % figuur 1
\fbox{\vbox to6pc{\hsize4cm\hfill\vfill}}
\caption{Спектральное разложение ряда данных по ВВП}
\label{penG}
\end{figure}

Sample of cross-reference to figure.
Figure~\ref{penG} shows that is not easy to get something on paper.



\section{Раздел (section)}

\subsection{Подраздел (subsection)}
Carr-Goldstein based their model on balancing methods and
biochemical know\-ledge. The original model (1980) contained an equation for the
oxygen dynamics which has been omitted in a second paper
(1981). This simplified model shall be discussed here.

\subsubsection{Подподраздел (subsubsection)}
Carr-Goldstein
based their model on balancing methods and
biochemical know\-ledge. The original model (1980) contained an equation for the
oxygen dynamics which has been omitted in a second paper
(1981). This simplified model shall be discussed here.

\section{Уравнения и прочая}

Два уравнения:
\begin{equation}
    C_{s}  =  K_{M} \frac{\mu/\mu_{x}}{1-\mu/\mu_{x}} \label{ccs}
\end{equation}

и

\begin{equation}
    G = \frac{P_{\rm opt} - P_{\rm ref}}{P_{\rm ref}} \mbox{\ }100 \mbox{\ }(\%)
\end{equation}

Две системы уравнений:

\begin{eqnarray}
  \frac{dS}{dt} & = & - \sigma X + s_{F} F\\
  \frac{dX}{dt} & = &   \mu    X\\
  \frac{dP}{dt} & = &   \pi    X - k_{h} P\\
  \frac{dV}{dt} & = &   F
\end{eqnarray}

и

\begin{eqnarray}
 \mu_{\rm substr} & = & \mu_{x} \frac{C_{s}}{K_{x}C_{x}+C_{s}}  \\
 \mu              & = & \mu_{\rm substr} - Y_{x/s}(1-H(C_{s}))(m_{s}+\pi /Y_{p/s}) \\
 \sigma           & = & \mu_{\rm substr}/Y_{x/s}+ H(C_{s}) (m_{s}+ \pi /Y_{p/s})
\end{eqnarray}

\printbibliography

\appendix

\section{Приложение}


We consider a sequence of queueing systems
indexed by $n$.  It is assumed that each system
is composed of $J$ stations, indexed by $1$
through $J$, and $K$ customer classes, indexed
by $1$ through $K$.  Each customer class
has a fixed route through the network of
stations.  Customers in class
$k$, $k=1,\ldots,K$, arrive to the
system according to a
renewal process, independently of the arrivals
of the other customer classes.  These customers
move through the network, never visiting a station
more than once, until they eventually exit
the system.


\subsection{Подраздел приложения}

However, different customer classes may visit
stations in different orders; the system
is not necessarily ``feed-forward.''
We define the {\em path of class $k$ customers} in
as the sequence of servers
they encounter along their way through the network
and denote it by
\begin{equation}
\mathcal{P}=\bigl(j_{k,1},j_{k,2},\dots,j_{k,m(k)}\bigr). \label{eq:path}
\end{equation}

Ссылки внутри документа: формуда to the formula \ref{eq:path} в приложении Appendix \ref{pril}.

\section*{Благодарности}
В этом разделе мы выражаем благодарности. Он не нумеруется и не добавляется в оглавление "--- для этого используется команда \verb"\section*". Большое спасибо Vytas Statulevicius, VTeX, Lithuania и Эконометрическому обществу США за идеи для данного стилевого файла.



\end{document}
