\documentclass[final,pdftex]{../../template/epsilonj}

\begin{document}
	
	% \microtypesetup{protrusion=false, expansion=false}
	\begin{frontmatter}
		\title{Кто выигрывает рэп-баттлы?}
		\runtitle{Кто выигрывает рэп-баттлы?}
		
		\begin{aug}
			\author{\imya{Андрей} \fam{Костырка}}%
			
			\runauthor{Костырка А. В.}
			
			\address{НИУ ВШЭ, Москва.}
		\end{aug}
		
		\begin{abstract}
			Теория хаоса "--- 
		\end{abstract}
		
		\begin{keyword}
			\kwd{рэп}
			\kwd{баттл}
			\kwd{жеребьёвка}
		\end{keyword}
		
	\end{frontmatter}
	
	% \microtypesetup{protrusion=true, expansion=true}
	
	\section{Влияние позиции участника на результат}

Методологически полезно отметить, что от выбора начальной точки сильно зависит результат эксперимента. К примеру, если симуляции проводятся посредством расчёта полной траектории после небольшого случайного сдвига в районе области высокой плотности аттрактора, через которую проходит множество траекторий, то результат будет давать хорошее представление о природе данных (рис. ***). Однако если в качестве начальной точки взять координату, через которую не проходит ни одна траектория аттрактора, и немного её двигать, то тогда результирующие траектории будут в начале пути очень похожи, объектам симуляции будет присущ один и тот же случайный артефакт, и распределение оценок корреляционной размерности будет неестественным и далёким от истины.

Пример. Рассмотрим аттрактор Руклиджа. 0,0,0  плохо, 0,0,5 "--- хорошо.



\end{document}