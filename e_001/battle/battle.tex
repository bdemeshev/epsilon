\documentclass[final,pdftex]{../../template/epsilonj}

\RequirePackage{graphicx}
\usepackage{microtype}

\RequirePackage[colorlinks,citecolor=blue,urlcolor=blue]{hyperref}

\newcommand{\specialcell}[2][c]{\begin{tabular}[#1]{@{}c@{}}#2\end{tabular}}

\begin{document}

% \microtypesetup{protrusion=false, expansion=false}
\begin{frontmatter}
\title{Кто выигрывает рэп-баттлы?}
\runtitle{Кто выигрывает рэп-баттлы?}

\begin{aug}
\author{\imya{Андрей} \fam{Зубанов}}%

\runauthor{А.~Зубанов}

\address{НИУ ВШЭ, Москва.}
\end{aug}

\begin{abstract}
	Статистика является мощным инструментом при изучении не только экономических и~социальных, но и~культурных явлений. Даже простые статистические методы могут оказаться полезными при изучении таких событий, как различные соревнования, телешоу, прослушивания.
\end{abstract}

\begin{keyword}
	\kwd{рэп}
	\kwd{баттл}
	\kwd{жеребьёвка}
\end{keyword}

\end{frontmatter}

% \microtypesetup{protrusion=true, expansion=true}

\section{Влияние позиции участника на результат}

Данная статья анализирует рэп-баттлы на примере популярного российского «Versus Battle». В~исследование вошли данные, собранные по выпускам «Первый сезон», «Второй сезон» и~«Межсезонье». В~баттле участвуют два человека (реже "--- две команды). Сущность рэп"=баттла «Versus» состоит в~том, чтобы посредством заранее подготовленного речитатива высказаться о~себе и~своём сопернике. Баттл проходит без музыкального сопровождения. Победитель определяется тремя судьями простым большинством голосов. На протяжении рассматриваемых баттлов участники стоят лицом друг к~другу, сбоку от них стоят судьи, вокруг располагаются немногочисленные болельщики, приглашённые на мероприятие. Баттл состоит из трёх раундов, каждый из которых начинает участник, выбранный жеребьёвкой первым.

\begin{figure}[htbp]
	\centering
	\includegraphics[width=8cm]{maxresdefault.jpg}
	\caption{Рэп-баттл}
\end{figure}

\begin{table}[htb] \centering
		\begin{tabular}{|c|c|c|}
			\toprule
			\specialcell{Победитель \\ по номеру} & \specialcell{Положение  первого \\ относительно судей} & \specialcell{Выиграл ли \\ участник справа} \\
			\midrule
			1& справа&1\\
			1& слева&0\\
			2& слева&1\\
			2& слева&1\\
			1& слева&0\\
			1& слева&0\\
			2& слева&1\\
			2& слева&1\\
			2& слева&1\\
			2& слева&1\\
			2& справа&0\\
			2& справа&0\\
			2& слева&1\\
			2& слева&1\\
			2& слева&1\\
			2& слева&1\\
			2& слева&1\\
			2& справа&0\\
			2& справа&0\\
			1& справа&1\\
			2& слева&1\\
			2& слева&1\\
			1& слева&0\\
			2& слева&1\\
			2& справа&0\\
			2& слева&1\\
			2& справа&0\\
			2& слева&1\\
\bottomrule			
		\end{tabular}
		\caption{Данные о~победах первого/второго по очереди участника и~участника, стоящего слева/справа от судей}
	\end{table}

Согласно данным, всего победил первый по очереди игрок 6~раз, второй игрок "--- 22~раза. Всего победил игрок слева 18~раз, справа "--- 10~раз. 

Основной гипотезой исследования является гипотеза о~том, что второй участник вследствие очерёдности хода имеет преимущество, предположительно из-за возможности ответить в~своём выступлении на выступление соперника. Также это возможно и~в~силу того, что выступающий последним лучше запоминается судьям и~поэтому выше оценивается. Вторая гипотеза "--- предположение о~том, что из-за своего положения относительно судей участник слева имеет преимущество.

Если результат баттла не зависит от очерёдности, то при любом назначенном номере участник должен выигрывать с~вероятностью, близкой к~$1/2$. Проверим эту гипотезу, учитывая, что номер выигравшего участника "--- случайная величина, распределённая биномиально.
\begin{equation}
	\frac{|\hat p_n-p_0|\sqrt{n}}{\sqrt{\hat p_n(1-\hat p_n)}}\sim \Studt_{28}
\end{equation}

Доля выигравших вторых игроков составляет $22/28$, всего наблюдений в~выборке~28. Значит,
\begin{equation}
z_{\text{расч}}=\frac{|22/28-1/2|\sqrt{28}}{\sqrt{22/28 \cdot (1-22/28)}}=8{,}98 > 2{,}05=\Studt_{0{,}975; 28}
\end{equation}

Таким образом, выступающие вторыми участники значимо чаще выигрывают рэп"=баттлы.

Также и~участники слева чаще выигрывают баттлы (вероятность $18/28$).
\begin{equation}
z_{\text{расч}}=\frac{|18/28-1/2|\sqrt{28}}{\sqrt{18/28 \cdot (1-18/28)}}=3{,}29>2{,}05=\Studt_{0{,}975; 28} 
\end{equation}
Однако при уровне значимости в~$0{,}001$ критическая статистика составляет $\Studt_{0{,}0005; 28}=3{,}67$, и~значимого различия не выявляется.

\primred{Кто берёт такие уровни значимости? Зачем?}

\section{Выводы}

Вопрос о~зависимости между позицией участника и~очерёдностью остаётся открытым, однако первое не должно влиять на второе, так как расстановка происходит после жеребьёвки.

Несмотря на то что жеребьёвка не показывается зрителю, а~организаторы не раскрывают процедуры, допустимо считать её случайной, так как это соответствует традициям рэп"=баттлов.

В~итоге нельзя с~уверенностью сказать о~причинно"=следственных связях между очерёдностью участников в~баттлах и~их результатами, ведь может существовать третий фактор, влияющий и~на то, и~на другое. Однако можно утверждать, что вторые участники действительно значимо чаще выигрывают рэп"=баттлы.

\vfill

\begin{center}
		\includegraphics[width=2cm]{versus.jpg}
\end{center}



\end{document}
