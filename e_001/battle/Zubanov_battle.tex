\documentclass[a4paper]{article}
\usepackage{mathtext} % русские буквы в формулах
\usepackage[X2, T2A]{fontenc} % шрифты Лапко-Ходулева (стандарт для СНГ)
\usepackage[english,russian]{babel} % русско-английские автопереносы и стандарты
\usepackage[utf8]{inputenc}         % кодировка исходного текста
\usepackage{euscript}
\usepackage{cmap}          % русский поиск в pdf
\usepackage{amsmath} %Математические окружения AMS
\usepackage{amsfonts} %Шрифты AMS
 \usepackage{amssymb} % Символы AMS
  \usepackage{mathtext} % Русские буквы в фомулах
  \usepackage{graphicx} % Вставить pdf- или png-файлы
%\usepackage[left=2cm,right=1.5cm,top=2cm,bottom=2cm]{geometry}
\usepackage{mathrsfs} % Красивый шрифт
  \usepackage{longtable}  % Длинные таблицы
  \usepackage{multirow} % Слияние строкв таблице
 \usepackage{indentfirst} % Отступ в первом абзаце.
\usepackage{wasysym}
 \newcommand*{\hm}[1]{#1\nobreak\discretionary{}%
            {\hbox{$\mathsurround=0pt #1$}}{}}
% точка в подписях рисунков и таблиц: вместо "Рис:" печатаем "Рис."
%\renewcommand{\captionlabeldelim}{.}
% Переопределение le ge в русских традициях
\renewcommand{\le}{\leqslant}
%\renewcommand{\leq}{\leqslant}
\renewcommand{\ge}{\geqslant}
%\renewcommand{\geq}{\geqslant}
% Переопределение kappa epsilon phi на русский лад
\renewcommand{\kappa}{\varkappa}
\renewcommand{\epsilon}{\varepsilon}
\renewcommand{\phi}{\varphi}
\renewcommand{\Pr}{\mathbb{P}}
\newcommand{\Ev}{\mathbb{E}}
% Прямые интегралы
%% Определяем свой шрифт "lsumb"
\DeclareSymbolFont{lsymb}{U}{euex}{m}{n}

\newcommand{\intl}{\int\limits}
\newcommand{\iintl}{\iint\limits}
\newcommand{\Var}{\mathrm{Var}}
\newcommand{\cov}{\mathrm{cov}}
\usepackage{misccorr}
\usepackage{multicol}
\usepackage{amsthm} % Расширенные теоремы

\theoremstyle{plain} %Это стиль по умолчанию, его можно не переопределять.
\newtheorem{theorem}{Теорема}[section]
\newtheorem{proposition}[theorem]{Утверждение}

\theoremstyle{definition}
\newtheorem{corollary}{Следствие}[theorem]
\newtheorem{problem}{Задача}
\newtheorem{problem*}{Задача}
\theoremstyle{remark}
\newtheorem*{nonum}{Решение}
\usepackage[space]{grffile}
\newcommand{\reminder}[1]
{ [[[ {\bf\marginpar{\mbox{$<==$}} #1 } ]]] }
\usepackage{comment}
%Ссылки
\usepackage{hyperref}

%Цвет
%\usepackage{color}
%\usepackage{pstricks-add}
%\usepackage{pgf,tikz}
%\usetikzlibrary{arrows}

\usepackage{tikz}
\usetikzlibrary{calc}
%\usepackage[dvipsnames]{xcolor}
\usepackage{epstopdf}
\epstopdfsetup{outdir=./}

\title{Кто выигрывает рэп-баттлы?}
\author{Андрей Зубанов}
\date{}

\begin{document}
\maketitle

Статистика является мощным инструментом не только при изучении экономических и социальных, но и культурных явлений. Даже простые статистические методы могут оказаться полезными при изучении таких событий, как различные соревнования, телешоу, прослушивания.

Данная статья анализирует рэп-баттлы на примере популярного российского Versus Battle. В исследование вошли данные, собранные по выпускам Versus Battle Первый сезон, Второй сезон, Межсезонье. В баттле участвует два человека (реже --- две команды). Сущность рэп-баттла Versus --- путём заранее подготовленного речетатива высказаться о себе и своём сопернике. Баттл проходит без музыккального сопровождения. Победитель определяется тремя судьями простым большинством голосов. На протяжении изучаемых баттлов участники стоят лицом друг к другу, сбоку от них стоят судьи, вокруг располагаются немногочисленные болельщики, приглашённые на мероприятие. Баттл состоит из трёх раундов, каждый из которых начинает участник, выбранный жеребьёвкой первым.


\vspace{1cm}

\begin{center}
\includegraphics[width=12cm]{maxresdefault.jpg}
\end{center}



\newpage
\begin{center}
\begin{table}
\begin{tabular}{|c|p{4cm}|p{3cm}|}
\hline
Победитель по номеру	&Положение первого относительно судей&Выиграл ли участник справа\\
\hline
1&	справа&1\\
1&	слева&0\\
2&	слева&1\\
2&	слева&1\\
1&	слева&0\\
1&	слева&0\\
2&	слева&1\\
2&	слева&1\\
2&	слева&1\\
2&	слева&1\\
2&	справа&0\\
2&	справа&0\\
2&	слева&1\\
2&	слева&1\\
2&	слева&1\\
2&	слева&1\\
2&	слева&1\\
2&	справа&0\\
2&	справа&0\\
1&	справа&1\\
2&	слева&1\\
2&	слева&1\\
1&	слева&0\\
2&	слева&1\\
2&	справа&0\\
2&	слева&1\\
2&	справа&0\\
2&	слева&1\\
\hline

\end{tabular}
\caption{Данные о победах первого/второго по очереди участника и участника, стоящего слева/справа от судей}
\end{table}
\end{center}

В соответствии с данными, всего победил 1 по очереди игрок --- 6 раз, 2 игрок --- 22 раза. Всего победил игрок слева --- 18 раз, справа --- 10 раз. 




Основной гипотезой исследования является гипотеза о том, что второй участник вследствие очерёдности хода имеет преимущество, предположительно из-за возможности ответить в своём выступлении на выступление соперника. Также это возможно из-за того, что выступающий последним лучше запоминается судьям и поэтому выше оценивается. Вторая гипотеза --- предположение о том, что из-за своего положения относительно судей участник слева имеет преимущество.

Если результат баттла не зависит от очерёдности, то каждый по номеру участник должен был выигрывать с вероятностью близкой к 1/2. Проверим эту гипотезу, учитывая, что номер выигравшего участника --- случайная величина, распределённая биномиально.

$$\frac{|\hat p_n-p_0|\sqrt{n}}{\sqrt{\hat p_n(1-\hat p_n)}}\sim t_{28}$$

Доля выигравших вторых игроков составляет 22/28, всего наблюдений в выборке --- 28. Значит,
$$Z_{расч}=\frac{|22/28-1/2|\sqrt{28}}{\sqrt{22/28(1-22/28)}}=8.98>2.05=t_{0.975, 28} $$

Таким образом, выступающие вторыми участники значимо чаще выигрывают рэп-баттлы.

Также и участники слева чаще выигрывают баттлы (вероятность 18/28).

$$Z_{расч}=\frac{|18/28-1/2|\sqrt{28}}{\sqrt{18/28(1-18/28)}}=3.29>2.05=t_{0.975,28} $$
однако при уровне значимости в $0.001$, $t_{0.001,28}=3.67$, и значимого различия нет.

Вопрос о зависимости между позицией участника и очередностью остаётся открытым, однако первое не должно влиять на второе, так как расстановка происходит после жеребьёвки.

Несмотря на то, что жеребьёвка не показывается зрителю, а организаторы не раскрывают процедуры, допустимо считать её случайной, так как это в традициях рэп-баттлов.

В итоге, нельзя с уверенностью сказать о причинно-следственных связях между очерёдностью участников в баттлах и их результатами, ведь может существовать третий фактор, влияющий и на то, и на другое. Однако можно утверждать, что вторые участники действительно значимо чаще выигрывают рэп-баттлы.

\vspace{1cm}
\begin{center}
\includegraphics[scale=.25]{versus.jpg}
\end{center}



\end{document}











