\documentclass{article}
\usepackage{pdfpages}

\usepackage[utf8]{inputenc}
\usepackage[russian]{babel}

\newcommand{\epsilonpdf}[2]{\includepdf[pages=-,
addtotoc={1, section, 1, #2, p:#1}]{#1/#1.pdf}}
% синтаксис:
% \epsilonpdf{название файла}{название в оглавлении}
% запятая в названии запрещена пока что

\newcommand{\epsiloncover}[1]{\includepdf[pages=-]{#1/#1.pdf}}



\begin{document}

\epsiloncover{cover}

\thispagestyle{empty}
\tableofcontents

\epsilonpdf{foreword}{Вступительное слово}
\epsilonpdf{functional-form}{Кирилл Фурманов «Как выбрать функциональную форму уравнения регрессии?»}
\epsilonpdf{metrics-errors}{Кирилл Фурманов; Борис Демешев  «Эконометрика: типичные ошибки студентов и аспирантов» }
\epsilonpdf{knn}{Саша Кузнецова «Метод k ближайших соседей и распознавание рукописных цифр»}
\epsilonpdf{battle}{Андрей Зубанов «Кто выигрывает рэп-баттлы?»}
\epsilonpdf{fun}{Весёлый уголок: Гальтон и ПДД}
\epsilonpdf{qa}{Андрей Костырка «Новости CrossValidated: выпуск 1»}
\epsilonpdf{qr-code}{Настя Игнатьева «QR-код или немного дополненной реальности»}
\epsilonpdf{season}{Иван Станкевич «Пара слов о методах сезонной корректировки»}
\epsilonpdf{problems}{Пять задачек}
\epsilonpdf{stol_zakazov}{Стол заказов}
\epsilonpdf{last_page}{Над номером работали}

\end{document}
