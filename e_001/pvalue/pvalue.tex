\documentclass[final,pdftex]{../../template/epsilonj}

\RequirePackage{graphicx}
\usepackage{microtype}

\RequirePackage[colorlinks,citecolor=blue,urlcolor=blue]{hyperref}

\usepackage{colortbl}
\definecolor{rrow}{rgb}{1,0.9,1}
\definecolor{ccol}{rgb}{1,1,0.6}
\definecolor{inters}{rgb}{1,0.9,0.6}

\addbibresource{../../template/epsilon.bib}

\begin{document}
	
	% \microtypesetup{protrusion=false, expansion=false}
	\begin{frontmatter}
		\title{\protect{$p$}-value: то, что вы всегда хотели узнать, но боялись спросить}
		\runtitle{\textit{p}-value: то, что вы всегда хотели узнать}
		
		\begin{aug}
			\author{\imya{Мария} \fam{Лысюк}}
			\runauthor{М. Лысюк}
			\address{НИУ ВШЭ, Москва.}
		\end{aug}
		
		\begin{abstract}
			Аннотация должна передавать краткое содержание работы.
			Она должна быть ясной, содержательной, релевантной и~короткой
			(не более 150~слов). Аннотация должна содержать информацию,
			необходимую для поиска по базам научных работ.
			В~аннотации не должно быть математических формул.
		\end{abstract}
		
		\begin{keyword}
			\kwd{p-value}
			\kwd{уровень значимости}
			\kwd{гипотезы}
			\kwd{интерпретация}
		\end{keyword}
		
	\end{frontmatter}
	
	% \microtypesetup{protrusion=true, expansion=true}


С~завидным постоянством хотя бы раз в~жизни студент, слушающий курс статистики, сталкивается с~вопросом экзаменатора (который обычно ещё надеется, что вопрос очевиден и~«вытягивает» с~помощью него студента): «Мистер $X$, что показывает $p$-value?»

И~тут для многих наступает этот неловкий момент, и~лицо выглядит примерно вот так (Эдвард Мунк, видимо, тоже не знал):

\begin{figure}[htbp]
	\centering
	\includegraphics[width=8cm]{munk-small.jpg}
	\caption{Лицо обычного человека, у~которого спросили, что такое $p$-value}
\end{figure}

Для того чтобы осознать сие, безусловно, великое понятие, мы должны, как Будда, пройти 7~ступеней познания. Как водится, примеры красноречивее всего доносят нужную информацию до мозга, так что поговорим сегодня про машинки.

Вкратце \textit{о~ходе эксперимента}. Мы будем узнавать, существует ли какая-либо зависимость между штрафом за лихачество водителя и цветом его машины. Гипотеза $\hypo_0$ будет выглядеть следующим образом.

\begin{itemize}
	\item $\hypo_0$: Выдача штрафа не зависит от цвета машины.
	\item $\hypo_1$: Водители с~красными машинами чаще получают штрафы за превышение скорости по сравнению с~синими машинами.
\end{itemize}


Итак, в~добрый путь!

\section{Семь ступеней познания $p$-value}
\textbf{Ступень 1. Выберите уровень значимости.} Начнём со~знакомого до боли. Строго говоря, уровень значимости "--- это мера, которая отражает наше предпочтение точности результатов: низкие уровни значимости говорят о~маленькой вероятности того, что полученные экспериментальным путём результаты случайны, и~наоборот. Согласно негласной конвенции, обычно используется 5\%-й уровень значимости. Это означает, что вероятность того, что наши результаты случайны, равна 0{,}05, а вероятность того, что мы сами повлияли на результат, равна~0{,}95.

\begin{itemize}
	\item \textit{Пример.} Возьмём и~мы уровень значимости в~5\,\%.
\end{itemize}

\textbf{Ступень 2. Определите ожидаемые результаты эксперимента.} Как правило, учёные, проводя эксперимент и~наблюдая впоследствии результаты, имеют представление о~том, какие результаты являются «типичными» до начала эксперимента. Это может быть основано на результатах из прошлых исследований, достоверных источников, научной литературы и~т.\,д. Для вашего эксперимента определите ваши ожидаемые результаты любым из способов.

\begin{itemize}
	\item \textit{Пример.} Пусть предыдущие исследования показали, что штрафы за превышение скорости чаще получают водители красных машин по сравнению с~синими. Также пусть результаты по всей стране показывают превышение красными в~отношении $2:1$ по сравнению с~синими. Мы же хотим узнать, применимы ли результаты, характерные для всей страны, к~нашему городу. Если мы возьмём случайную выборку из 150~машинок, которым выписали штрафы, мы будем ожидать, что 100 машин будут красными, а~50 "--- синими, \emph{если наша полиция выписывает штрафы согласно национальной тенденции}.
	
\end{itemize}

\begin{figure}[htbp]
	\centering \Large
 \begin{tabular}{|c|c|} \toprule
 	{\color{red} Красная машинка} & {\color{blue} Синяя машинка} \\	\midrule
 	{\color{red} \huge \textbf{100}} & {\color{blue} \huge \textbf{50}} \\
 	\bottomrule
 \end{tabular}
	\caption{Ожидаемые значения количества штрафов}
\end{figure}

\textbf{Ступень 3. Определите наблюдаемые результаты эксперимента.} После того как мы определили ожидаемые результаты, проводим реальный эксперимент и~получаем наблюдаемые результаты. Если мы каким-либо образом повлияли и~наблюдаемые результаты отличаются от ожидаемых, возможна одна из двух ситуаций:
\begin{enumerate}
	\item Это произошло случайно.
	\item Те условия, в~которых мы проводили эксперимент,\emph{повлияли} на исход.
\end{enumerate}

Как правило, цель нахождения $p$-value "--- определить, правда ли, что наблюдаемые результаты отличаются от ожидаемых настолько, что мы не можем отвергнуть нулевую гипотезу (гипотезу о~том, что нет связи между переменными и~наблюдаемым результатом).

\primred{Что значит «определить наблюдаемые результаты»? Это как?}

\begin{itemize}
	\item \textit{Пример.} Пусть в~нашем городе мы произвольно выбрали 150~красных и~синих машин нарушителей. Оказалось, что 90~штрафов выписали красным машинам, а~60 "--- синим. Это отличается от ожидаемых 100 и~50 соответственно. Правда ли, что те условия, в~которых мы проводили эксперимент (в~нашем случае смена источника данных с~национальных на местные) послужила причиной изменения результатов, или действия городской полиции так~же смещены, как и~предсказывает национальная средняя оценка, и~мы просто наблюдаем случайную вариацию? $p$"=значение спешит на помощь!
\end{itemize}

\begin{figure}[htbp]
	\centering \Large
	\begin{tabular}{|c|c|} \toprule
		{\color{red} Красная машинка} & {\color{blue} Синяя машинка} \\	\midrule
		{\color{red} \huge \textbf{90}} & {\color{blue} \huge \textbf{60}} \\
		\bottomrule
	\end{tabular}
	\caption{Наблюдаемые количества штрафов}
\end{figure}

\textbf{Ступень 4. Определите степени свободы в~вашем эксперименте.} Степени свободы отражают меру изменчивости, характерную для исследования, которая определяется количеством переменных, которые вы изучаете. Степени свободы определяются как $n-1$, где $n$ "--- это количество переменных, используемых в~эксперименте.

\primred{Что это за misdirection? Степени свободы чего? В регрессии для Residual S.\,S. это, например, количество наблюдений минус количество параметров. Не лучше ли дать более общее и понятное определение? }

\begin{itemize}
	\item \textit{Пример.} У~нас есть две переменные: количество красных машин и~количество синих машин. Поэтому степеней свободы всего $2-1=1$, т.\,е. одна.
\end{itemize}

\textbf{Ступень 5. Сравните наблюдаемые результаты с~ожидаемыми с~помощью распределения $\chi^2$.} $\chi^2$ "--- статистика, численно измеряющая разницу между ожидаемыми и~наблюдаемыми результатами. Уравнение:
\[
\chi^2=\sum_{i=0}^{n} \frac {(h_i-e_i)^2}{e_i},
\]

где $h$ "--- значение наблюдаемой переменной, а~$e$ "--- ожидаемой.

\begin{itemize}
	\item \textit{Пример.} Мы должны просуммировать значения для всех возможных переменных, то есть в~нашем случае для синих и~красных машинок:
\end{itemize}
\[
\chi^2=\sum_{i=0}^{1} \frac {(h_i-e_i)^2}{e_i} = \frac{(90-100)^2}{100} + \frac{(60-50)^2}{50} = \frac{(-10)^2}{100} + \frac{10^2}{50} = 1 + 2 = \fbox{3}.
\]


\textbf{Ступень 6. Используем таблицу $\chi2$"=распределения, чтобы аппроксимировать $p$-value.} Скрестила пальцы: надеюсь, что все умеют пользоваться таблицами распределений.

\begin{itemize}
	\item \textit{Пример.} Наше значение статистики $\chi^2$ равно~3. Далее пользуемся таблицей~\ref{tab:chisq} для нахождения $p$-значения. У~нас одна степень свободы (\ENGs{degree of freedom}), поэтому берём первую строку и~ищем там первое значение, превышающее значение нашего $\chi^2=3$. Оно равно~3{,}84. Соответствующее $p$-значение равно 0{,}05. Это означает, что наше $p$-value располагается между 0{,}05 и~0{,}1.
\end{itemize}

\begin{table}[hbt]
	\begin{tabular}{|r|rrrrrrrrr|} \toprule
		& \multicolumn{9}{c|}{$p$-value} \\
		df  & 20\%  & 10\%  & \cellcolor{ccol} 5\%   & 2{,}5\% & 1\%   & 0{,}5\% & 0{,}25\% & 0{,}1\% & 0{,}05\% \\ \cmidrule{2-10}
	\cellcolor{rrow}	1  & \cellcolor{rrow} 1{,}64  & \cellcolor{rrow} 2{,}71  & \cellcolor{inters} 3{,}84  & \cellcolor{rrow} 5{,}02  & \cellcolor{rrow} 6{,}63  & \cellcolor{rrow} 7{,}88  & \cellcolor{rrow} 9{,}14   & \cellcolor{rrow} 10{,}83 & \cellcolor{rrow} 12{,}12  \\
		2  & 3{,}22  & 4{,}61  & \cellcolor{ccol} 5{,}99  & 7{,}38  & 9{,}21  & 10{,}60 & 11{,}98  & 13{,}82 & 15{,}20  \\
		3  & 4{,}64  & 6{,}25  & \cellcolor{ccol} 7{,}81  & 9{,}35  & 11{,}34 & 12{,}84 & 14{,}32  & 16{,}27 & 17{,}73  \\
		4  & 5{,}99  & 7{,}78  & \cellcolor{ccol} 9{,}49  & 11{,}14 & 13{,}28 & 14{,}86 & 16{,}42  & 18{,}47 & 20{,}00  \\
		5  & 7{,}29  & 9{,}24  & \cellcolor{ccol} 11{,}07 & 12{,}83 & 15{,}09 & 16{,}75 & 18{,}39  & 20{,}52 & 22{,}11  \\
		10 & 13{,}44 & 15{,}99 & \cellcolor{ccol} 18{,}31 & 20{,}48 & 23{,}21 & 25{,}19 & 27{,}11  & 29{,}59 & 31{,}42  \\
		20 & 25{,}04 & 28{,}41 & \cellcolor{ccol} 31{,}41 & 34{,}17 & 37{,}57 & 40{,}00 & 42{,}34  & 45{,}31 & 47{,}50  \\
		30 & 36{,}25 & 40{,}26 & \cellcolor{ccol} 43{,}77 & 46{,}98 & 50{,}89 & 53{,}67 & 56{,}33  & 59{,}70 & 62{,}16  \\
		40 & 47{,}27 & 51{,}81 & \cellcolor{ccol} 55{,}76 & 59{,}34 & 63{,}69 & 66{,}77 & 69{,}70  & 73{,}40 & 76{,}09  \\
		50 & 58{,}16 & 63{,}17 & \cellcolor{ccol} 67{,}50 & 71{,}42 & 76{,}15 & 79{,}49 & 82{,}66  & 86{,}66 & 89{,}56  \\ \bottomrule
	\end{tabular}
	\caption{Критические статистики для распределения $\chi^2$} \label{tab:chisq}
\end{table}



\textbf{Ступень 7. Вот мы и~добрались до конца! Осталось решить, отвергается или нет нулевая гипотеза.} Если $p$-value меньше, чем уровень значимости, то мои поздравления, можете отсылать вашу работу в~топовые журналы! Вы доказали, что высока вероятность того, что есть значимая корреляция между переменными, которыми вы манипулируете, и~наблюдаемыми результатами. Если ли же $p$-значение больше выбранного уровня значимости, вы не можете с~точностью сказать, случайны ли полученные вами результаты, или они являются результатом ваших действий.

\begin{itemize}
	\item \textit{Пример.} Наше $p$-значение находится в~границах от 0{,}05 до 0{,}1. Это определённо меньше, чем выбранный уровень значимости, равный 0,05, поэтому, к~сожалению, мы не можем отвергнуть нулевую гипотезу. Другими словами, мы не достигли желаемого уровня в~95\,\%, чтобы с~точностью сказать, что в~нашем городе полиция выдаёт штрафы красным и~синим машинам в~пропорции, значительно отличающейся от национального уровня. Иначе говоря, есть вероятность 5--10\,\% того, что изменения в~выдаче штрафов красным и~синим машинам связаны не со~сменой локации, а~с~чистой случайностью. Ввиду того что мы ищем вероятность, меньшую, чем 0{,}05, мы не можем быть \emph{уверены}, что полиция нашего города более склонна выдавать штрафы красным машинам: есть маленькая, но статистически значимая вероятность того, что это не так.
\end{itemize}

А~теперь, после того как мы проделали такой до-олгий путь к~нирване, ввёдем, наконец, определение.

\textit{$p$-значение} "--- это вероятность того, что случайная величина с~данным распределением (распределением тестовой статистики при нулевой гипотезе) примет значение, не меньшее, чем фактическое значение тестовой статистики.

\nocite{wikihow14pvalue}
\nocite{labstatspvalue}
И~напоследок. Господин Гудман (\cite{goodman2008dirty}) написал чудную статью о~недопонимании $p$-value и~о~тех ошибках в~интерпретации, которые обычно допускают студенты. \textbf{Не делайте так! Опасно для жизни!} Помните:

\begin{itemize}
	\item $p=0{,}05$ не означает, что есть 5\%-я вероятность того, что нулевая гипотеза верна.
	\item $p=0{,}05$ не означает, что есть 5\%-я вероятность ошибки первого рода.
	\item $p=0{,}05$ не означает, что есть 95\%-я вероятность того,что результаты будут такими же при повторении эксперимента.
	\item $p > 0{,}05$ не означает, что нет разницы между наблюдаемыми переменными.
	\item $p < 0{,}05$ не означает, что нулевая гипотеза не отвергается.
\end{itemize}


\printbibliography	
	
\end{document}

