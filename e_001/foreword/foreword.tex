\documentclass[11pt]{article}

\usepackage{epsilonj}


% \usepackage[hyphens]{url}
% Лучший способ сделать избежать плохого переноса — это переписать текст. — А. В.

\RequirePackage[colorlinks,citecolor=blue,urlcolor=blue]{hyperref}



\begin{document}

\pagestyle{empty}

\begin{center}
\Large\textbf{Вступительное слово}
\end{center}


Замечательные, но пока ещё малознакомые читатели!

Вы, наверно, не раз заглядывали в~научные журналы и~могли заметить особенности академического языка. Статьи в~такие журналы принято писать обезличенно, в~страдательном залоге: было сделано, было проведено. Если вы ещё и~писали такие статьи, то знаете, как хочется иногда выражаться хоть немножко человечнее: «я сделал», «мы провели», «а давайте"=ка попробуем вот это». в~прошлом году я попробовал (весьма сдержанно "--- куда сдержаннее, чем пишу сейчас) и~получил ответ рецензента: стиль не соответствует нормам академического языка.

Ещё от научных статей требуют новизны. Например, если вы напишете про коэффициент ранговой корреляции Спирмена, то вам могут сказать: незачем, про это уже Спирмен написал. А если хочется?

Мы сделали журнал, в~котором можно опубликовать статью без строгого академического стиля и~научной новизны "--- лишь бы хорошая была. Как мы будем определять, хорошая ли статья? Попробуем как-нибудь разобраться. Если случилось так, что автор ошибся, мы можем предложить ему исправить ошибку, можем вставить своё примечание, а можем просто этого не заметить. Если заметите вы "--- пишите нам.

Собственно, статьи в~этом журнале вообще не обязаны быть научными: приветствуются и~методические, и~дидактические работы. Конечно, это чревато тем, что наши авторы будут терять умеренность и~писать всякую ерунду "--- так ведь чуточку побаловаться можно!

Архив журнала доступен на страничке \href{http://bdemeshev.github.io/epsilon/}{github.com/bdemeshev/epsilon}. Замечательные читатели, пишите нам смелее на почту \href{mailto:journal.epsilon@gmail.com}{journal.epsilon@gmail.com}! Предлагайте свои статьи, комментируйте чужие "--- мы будем очень рады :)!


\begin{flushright}
Кирилл Фурманов.
\end{flushright}


\end{document}
