\documentclass[final,pdftex]{../../template/epsilonj}

\RequirePackage{graphicx}
\RequirePackage[colorlinks,citecolor=blue,urlcolor=blue]{hyperref}

\addbibresource{../../template/epsilon.bib}

\begin{document}

\setcounter{page}{59}

% \microtypesetup{protrusion=false, expansion=false}
\begin{frontmatter}
\title{Стол заказов}
\runtitle{Стол заказов}

\begin{aug}
\author{\imya{Борис} \fam{Демешев}}%


%\runauthor{Борис Демешев}

%\address{НИУ ВШЭ, Москва.}
\end{aug}

%\begin{abstract}
%\end{abstract}

%\begin{keyword}
%	\kwd{статистика}
%	\kwd{вопросы}
%	\kwd{ответы}
%	\kwd{интернет}
%\end{keyword}

\end{frontmatter}

% \microtypesetup{protrusion=true, expansion=true}

% \section{xxx}


* пакет и текст по rlms
* пакет и текст по картам
* векторно-матричное дифференциирование с примерами 
примеры: оценка ковариационной матрицы ML, МНК, LDA, PCA, canonical correlation
основные случаи, немного упражнений (?)
* LDA по чесноку. честный вывод формул в LDA с примерами задач
* рассказ и решение группового тура от победителей
* про частную корреляцию с теоремой Фриша-Вау
* теорема Фриша-Вау с примерами (регрессия с и без константы, частная корреляция) и доказательством
* пакет по российским данным с helper-функциями: загрузка, список источников


Гид для будущих авторов

Мы принимаем статьи написанные с помощью \LaTeX, языка разметки markdown и грамотного программирования (literate programming) с использованием R (\code|Rmd|, \code|Rnw|). 


Смысловая часть


Техническая часть





\end{document}
